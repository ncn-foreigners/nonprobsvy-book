% Options for packages loaded elsewhere
\PassOptionsToPackage{unicode}{hyperref}
\PassOptionsToPackage{hyphens}{url}
\PassOptionsToPackage{dvipsnames,svgnames,x11names}{xcolor}
%
\documentclass[
  letterpaper,
  DIV=11,
  numbers=noendperiod]{scrreprt}

\usepackage{amsmath,amssymb}
\usepackage{lmodern}
\usepackage{iftex}
\ifPDFTeX
  \usepackage[T1]{fontenc}
  \usepackage[utf8]{inputenc}
  \usepackage{textcomp} % provide euro and other symbols
\else % if luatex or xetex
  \usepackage{unicode-math}
  \defaultfontfeatures{Scale=MatchLowercase}
  \defaultfontfeatures[\rmfamily]{Ligatures=TeX,Scale=1}
\fi
% Use upquote if available, for straight quotes in verbatim environments
\IfFileExists{upquote.sty}{\usepackage{upquote}}{}
\IfFileExists{microtype.sty}{% use microtype if available
  \usepackage[]{microtype}
  \UseMicrotypeSet[protrusion]{basicmath} % disable protrusion for tt fonts
}{}
\makeatletter
\@ifundefined{KOMAClassName}{% if non-KOMA class
  \IfFileExists{parskip.sty}{%
    \usepackage{parskip}
  }{% else
    \setlength{\parindent}{0pt}
    \setlength{\parskip}{6pt plus 2pt minus 1pt}}
}{% if KOMA class
  \KOMAoptions{parskip=half}}
\makeatother
\usepackage{xcolor}
\setlength{\emergencystretch}{3em} % prevent overfull lines
\setcounter{secnumdepth}{5}
% Make \paragraph and \subparagraph free-standing
\ifx\paragraph\undefined\else
  \let\oldparagraph\paragraph
  \renewcommand{\paragraph}[1]{\oldparagraph{#1}\mbox{}}
\fi
\ifx\subparagraph\undefined\else
  \let\oldsubparagraph\subparagraph
  \renewcommand{\subparagraph}[1]{\oldsubparagraph{#1}\mbox{}}
\fi


\providecommand{\tightlist}{%
  \setlength{\itemsep}{0pt}\setlength{\parskip}{0pt}}\usepackage{longtable,booktabs,array}
\usepackage{calc} % for calculating minipage widths
% Correct order of tables after \paragraph or \subparagraph
\usepackage{etoolbox}
\makeatletter
\patchcmd\longtable{\par}{\if@noskipsec\mbox{}\fi\par}{}{}
\makeatother
% Allow footnotes in longtable head/foot
\IfFileExists{footnotehyper.sty}{\usepackage{footnotehyper}}{\usepackage{footnote}}
\makesavenoteenv{longtable}
\usepackage{graphicx}
\makeatletter
\def\maxwidth{\ifdim\Gin@nat@width>\linewidth\linewidth\else\Gin@nat@width\fi}
\def\maxheight{\ifdim\Gin@nat@height>\textheight\textheight\else\Gin@nat@height\fi}
\makeatother
% Scale images if necessary, so that they will not overflow the page
% margins by default, and it is still possible to overwrite the defaults
% using explicit options in \includegraphics[width, height, ...]{}
\setkeys{Gin}{width=\maxwidth,height=\maxheight,keepaspectratio}
% Set default figure placement to htbp
\makeatletter
\def\fps@figure{htbp}
\makeatother
\newlength{\cslhangindent}
\setlength{\cslhangindent}{1.5em}
\newlength{\csllabelwidth}
\setlength{\csllabelwidth}{3em}
\newlength{\cslentryspacingunit} % times entry-spacing
\setlength{\cslentryspacingunit}{\parskip}
\newenvironment{CSLReferences}[2] % #1 hanging-ident, #2 entry spacing
 {% don't indent paragraphs
  \setlength{\parindent}{0pt}
  % turn on hanging indent if param 1 is 1
  \ifodd #1
  \let\oldpar\par
  \def\par{\hangindent=\cslhangindent\oldpar}
  \fi
  % set entry spacing
  \setlength{\parskip}{#2\cslentryspacingunit}
 }%
 {}
\usepackage{calc}
\newcommand{\CSLBlock}[1]{#1\hfill\break}
\newcommand{\CSLLeftMargin}[1]{\parbox[t]{\csllabelwidth}{#1}}
\newcommand{\CSLRightInline}[1]{\parbox[t]{\linewidth - \csllabelwidth}{#1}\break}
\newcommand{\CSLIndent}[1]{\hspace{\cslhangindent}#1}

\KOMAoption{captions}{tableheading}
\makeatletter
\makeatother
\makeatletter
\@ifpackageloaded{bookmark}{}{\usepackage{bookmark}}
\makeatother
\makeatletter
\@ifpackageloaded{caption}{}{\usepackage{caption}}
\AtBeginDocument{%
\ifdefined\contentsname
  \renewcommand*\contentsname{Table of contents}
\else
  \newcommand\contentsname{Table of contents}
\fi
\ifdefined\listfigurename
  \renewcommand*\listfigurename{List of Figures}
\else
  \newcommand\listfigurename{List of Figures}
\fi
\ifdefined\listtablename
  \renewcommand*\listtablename{List of Tables}
\else
  \newcommand\listtablename{List of Tables}
\fi
\ifdefined\figurename
  \renewcommand*\figurename{Figure}
\else
  \newcommand\figurename{Figure}
\fi
\ifdefined\tablename
  \renewcommand*\tablename{Table}
\else
  \newcommand\tablename{Table}
\fi
}
\@ifpackageloaded{float}{}{\usepackage{float}}
\floatstyle{ruled}
\@ifundefined{c@chapter}{\newfloat{codelisting}{h}{lop}}{\newfloat{codelisting}{h}{lop}[chapter]}
\floatname{codelisting}{Listing}
\newcommand*\listoflistings{\listof{codelisting}{List of Listings}}
\makeatother
\makeatletter
\@ifpackageloaded{caption}{}{\usepackage{caption}}
\@ifpackageloaded{subcaption}{}{\usepackage{subcaption}}
\makeatother
\makeatletter
\@ifpackageloaded{tcolorbox}{}{\usepackage[many]{tcolorbox}}
\makeatother
\makeatletter
\@ifundefined{shadecolor}{\definecolor{shadecolor}{rgb}{.97, .97, .97}}
\makeatother
\makeatletter
\makeatother
\ifLuaTeX
  \usepackage{selnolig}  % disable illegal ligatures
\fi
\IfFileExists{bookmark.sty}{\usepackage{bookmark}}{\usepackage{hyperref}}
\IfFileExists{xurl.sty}{\usepackage{xurl}}{} % add URL line breaks if available
\urlstyle{same} % disable monospaced font for URLs
\hypersetup{
  pdftitle={Modern inference methods for non-probability samples with R},
  pdfauthor={Łukasz Chrostowski, Maciej Beręsewicz},
  colorlinks=true,
  linkcolor={blue},
  filecolor={Maroon},
  citecolor={Blue},
  urlcolor={Blue},
  pdfcreator={LaTeX via pandoc}}

\title{Modern inference methods for non-probability samples with R}
\author{Łukasz Chrostowski, Maciej Beręsewicz}
\date{4/20/2023}

\begin{document}
\maketitle
\ifdefined\Shaded\renewenvironment{Shaded}{\begin{tcolorbox}[sharp corners, breakable, frame hidden, interior hidden, boxrule=0pt, borderline west={3pt}{0pt}{shadecolor}, enhanced]}{\end{tcolorbox}}\fi

\renewcommand*\contentsname{Table of contents}
{
\hypersetup{linkcolor=}
\setcounter{tocdepth}{2}
\tableofcontents
}
\bookmarksetup{startatroot}

\hypertarget{nomenclature}{%
\chapter{Nomenclature}\label{nomenclature}}

\begin{longtable}[]{@{}
  >{\raggedright\arraybackslash}p{(\columnwidth - 2\tabcolsep) * \real{0.2778}}
  >{\raggedright\arraybackslash}p{(\columnwidth - 2\tabcolsep) * \real{0.7222}}@{}}
\toprule()
\begin{minipage}[b]{\linewidth}\raggedright
Symbol
\end{minipage} & \begin{minipage}[b]{\linewidth}\raggedright
Meaning
\end{minipage} \\
\midrule()
\endhead
\(\mathbb{E}\) & Expected value operator \\
\(N\) & True population size \\
\(\mathcal{U}\) & Finite population with N units \\
\(y\) & Response variable \\
\(\bx\) & Auxiliary variables \\
\(p\) & Number of auxiliary variables \\
\(S_A\) & Nonprobablity sample \\
\(S_B\) & Probability sample \\
\(\pi_i^B\) & Probability of belonging to a probability sample for the
given unit \\
\(\pi_i^A\) & Probability of belonging to a non-probability sample for
the given unit \\
\(\dot{\pi}_i^A(\cdot)\) & Derivative of \(\pi_i^A\) at point
\(\cdot\) \\
\(d_i^B\) & Design weight for the given unit in a probability sample \\
\(n_A\) & Size of non-probability sample \\
\(n_B\) & Size of probability sample \\
\(\mu_{y}\) & Mean of population for response variable \\
\(R_{i}^A\) & An indicator function for nonprobability sample \\
\(\omega_i\) & Frequency weight \\
\(\ell\) & Log-likelihood function for the model \\
\(\hat{N}\) & Point estimate for true population size \\
\(\text{var}\) & Variance operator \\
\(\text{var}_p\) & Design-based variance operator under the probability
sampling design for \(S_B\) \\
\(Diag(\cdot)\) & A diagonal matrix constructed from vector \(\cdot\) \\
\(\btheta\) & Set of the parameters to estimate for selection model \\
\(\bbeta\) & Set of the parameters to estimate for outcome model \\
\(\phi(\cdot)\) & Probability density function of the normal
distribution in point \(\cdot\) \\
\(\Phi(\cdot)\) & Cumulative distribution function of the normal
distribution in point \(\cdot\) \\
\(\hat{\mu}_{MI}\) & Mass imputation estimator of population mean \\
\(\hat{\mu}_{IPW}\) & Inverse probability weighted estimator of
population mean \\
\(\hat{\mu}_{DR}\) & Doubly robust estimator of population mean \\
\(q\) & Model for selection mechanism for non-probability sample \\
\bottomrule()
\end{longtable}

\bookmarksetup{startatroot}

\hypertarget{introduction-and-overview}{%
\chapter{Introduction and Overview}\label{introduction-and-overview}}

\bookmarksetup{startatroot}

\hypertarget{inverse-probability-weighting}{%
\chapter{Inverse probability
weighting}\label{inverse-probability-weighting}}

\hypertarget{motivation-and-assumptions}{%
\section{Motivation and assumptions}\label{motivation-and-assumptions}}

Let \(\mathcal{U}=\{1,2, \ldots, N\}\) represent the finite population
with N units and
\(\left\{\left(\bx_i, y_i\right), i \in \mathcal{S}_{\mathrm{A}}\right\}\)
and \(\{\left(\bx_i, d_i^B), i \in \mathcal{S}_{\mathrm{B}}\right\}\) be
the datasets from non-probability and probability samples respectively.
Following assumptions are required for this model:

\begin{enumerate}
\def\labelenumi{\arabic{enumi}.}
\item
  The selection indicator \(R_i\) and the response variable \(y_i\) are
  independent given the set of covariates \(x_i\).
\item
  All units have a nonzero propensity score, that is, \(\pi_i^A > 0\)
  for all \(i\).
\item
  The indicator variables \(R_i^A\) and \(R_j^A\) are independent for
  given \(x_i\) and \(x_j\) for \(i \neq j\).
\end{enumerate}

\hypertarget{maximum-likelihood-estimation}{%
\section{Maximum likelihood
estimation}\label{maximum-likelihood-estimation}}

Suppose that propensity score can be modelled parametrically as
\(\mathbb{P}\left(R_i=1 \mid \bx_i\right) = \pi(\bx_{i}, \btheta_{0})\).
The maximum likelihood estimator is computed as
\(\hat{\pi}_{i}^{A} = \pi(\bx_{i}, \hat{\btheta}_{0})\), where
\(\hat{\btheta}_{0}\) is the maximizer of the following log-likelihood
function:

\[
    \begin{split}
 \ell(\boldsymbol{\theta}) & =\sum_{i=1}^N\left\{R_i \log \pi_i^{\mathrm{A}}+\left(1-R_i\right) \log \left(1-\pi_i^{\mathrm{A}}\right)\right\} \\ & =\sum_{i \in \mathcal{S}_{\mathrm{A}}} \log \left\{\frac{\pi\left(\boldsymbol{x}_i, \boldsymbol{\theta}\right)}{1-\pi\left(\boldsymbol{x}_i, \boldsymbol{\theta}\right)}\right\}+\sum_{i=1}^N \log \left\{1-\pi\left(\boldsymbol{x}_i, \boldsymbol{\theta}\right)\right\}
    \end{split}
\]

Since we do not observe \(\bx_i\) for all units, Yilin Chen, Pengfei Li
\& Changbao Wu presented following log-likelihood function is subject to
data integration basing on samples \(S_A\) and \(S_B\). They proposed
logistic regression model with
\(\pi(\bx_{i}, \btheta) = \frac{\exp(\bx_{i}^{\top}\btheta)}{\exp(\bx_{i}^{\top}\btheta) + 1}\)
in order to estimate \(\btheta\). We expanded this approach on probit
regression and complementary log-log model. For the sake of accuracy,
let us recall that the probit and cloglog models are based on the
assumption that model takes the form
\(\pi(\bx_{i},\btheta) = \Phi(\bx_{i}^{\top}\btheta)\) and
\(\pi(\bx_{i}, \btheta) = 1 - \exp(-\exp(\bx_{i}^{\top}\btheta))\)
respectively.

\[
    \ell^{*}(\btheta) = \sum_{i \in S_{A}}\log \left\{\frac{\pi(\bx_{i}, \btheta)}{1 - \pi(\bx_{i},\btheta)}\right\} + \sum_{i \in S_{B}}d_{i}^{B}\log\{1 - \pi({\bx_{i},\btheta})\}
\] In the following subsections we present the full derivation of the
MLE, depending on the assumed model for the propensity score.

\hypertarget{logistic-regression}{%
\subsection{Logistic regression}\label{logistic-regression}}

Log-likelihood function with logistic regression is given by \[
\ell^{*}(\btheta) = \sum_{i \in S_A}\bx_{i}^{\top}\btheta - \sum_{i \in S_B}d_{i}^{B}\log\{1 + \exp(\bx_{i}^{\top}\btheta)\}
\] with analytical gradient and hessian given by \[
\frac{\partial \ell^*}{\partial\btheta} = \sum_{i \in S_{A}}\bx_{i} - \sum_{i \in S_{B}}d_{i}^{B}\pi(\bx_{i}, \btheta)\bx_{i}
\] and \[
    \frac{\partial^{2} \ell^{*}}{\partial\btheta^{T} \partial\btheta} =- \sum_{i \in S_B}d_i^B\pi(\bx_i,\btheta)(1 - \pi(\bx_i,\btheta))\bx_i\bx_i^{\top} = \bX_B^{\top}\operatorname{\bW}_{B}\bX_B,
\] respectively, where \[
\begin{align*}
    \operatorname{\bW}_{B} =
    diag & \left(-d_1^B\pi(\bx_{1},\btheta)(1 - \pi(\bx_{1},\btheta)), -d_2^B\pi(\bx_{2},\btheta)(1 - \pi(\bx_{2},\btheta)), \right. \\
     & \left. \ldots, -d_{n_{B}}^{B}\pi(\bx_{n_{B}},\btheta)(1 - \pi(x_{n_{B}},\btheta))\right).
\end{align*}
\]

\hypertarget{complementary-log-log-regression}{%
\subsection{Complementary log-log
regression}\label{complementary-log-log-regression}}

Similarly, log-likelihood function has form of \[
\ell^{*}(\btheta) = \sum_{i \in S_{A}}\left\{\log\{1 - \exp(-\exp(\bx_{i}^{\top}\btheta))\} + \exp(\bx_{i}^{\top}\btheta)\right\} - \sum_{i \in S_{B}} d_{i}^{B}\exp(\bx_{i}^{\top}\btheta)
\] with analytical gradient and hessian equal to \[
    \frac{\partial \ell^*}{\partial\btheta} = \sum_{i \in S_{A}}\frac{\exp(\bx_{i}^{\top}\btheta)\bx_{i}}{\pi(\bx_{i}, \btheta)} - \sum_{i \in S_{B}}d_{i}^{B}\exp(\bx_{i}^{T}\btheta)\bx_{i}
\] and \[
    \begin{split}
    \frac{\partial^{2} \ell^{*}}{\partial\btheta^{T} \partial\btheta} & = \sum_{i \in S_A} \frac{\exp(\bx_{i}^{\top}\btheta)}{\pi(\bx_{i}, \btheta)} \left\{1 - \frac{\exp(\bx_{i}^{\top}\btheta)}{\pi(\bx_{i}, \btheta)} + \exp(\bx_{i}^{\top}\btheta)\right\}\bx_i\bx_i^{\top} - \sum_{i \in S_B}d_i^B\exp (\bx_{i}^{\top}\btheta)\bx_i\bx_i^{\top} \\ & = \bX_A^{\top}\operatorname{\bW}_{Ac}\bX_A - \bX_B^{\top}\operatorname{\bW}_{Bc}\bX_B,
    \end{split}
\] respectively, where \[
\begin{align*}
    \operatorname{\bW}_{Ac} =  Diag & \left(\frac{\exp(\bx_{1}^{\top}\btheta)}{\pi(\bx_{1}, \btheta)} \left\{1 - \frac{\exp(\bx_{1}^{\top}\btheta)}{\pi(\bx_{1}, \btheta)} + \exp(\bx_{1}^{\top}\btheta)\right\}, \right.
    \\
    & \left. \frac{\exp(\bx_{2}^{\top}\btheta)}{\pi(\bx_{2}, \btheta)} \left\{1 - \frac{\exp(\bx_{2}^{\top}\btheta)}{\pi(\bx_{2}, \btheta)} + \exp(\bx_{2}^{\top}\btheta)\right\}, \right.
    \\
    & \left. \ldots, \right.
    \\ 
    & \left. \frac{\exp(\bx_{n_A}^{\top}\btheta)} {\pi(\bx_{n_A}, \btheta)} \left\{1 - \frac{\exp(\bx_{n_A}^{\top}\btheta)}{\pi(\bx_{n_A}, \btheta)} + \exp(\bx_{n_A}^{\top}\btheta)\right\} \right)
\end{align*}
\] and \[
\begin{align*}
    \operatorname{\bW}_{Bc} = Diag \left(d_1^B\exp (\bx_{1}^{\top}\btheta), d_2^B\exp (\bx_{2}^{\top}\btheta), \ldots, d_{n_B}^B\exp (\bx_{n_{B}}^{\top}\btheta)\right).
\end{align*}
\]

\hypertarget{probit-regression}{%
\subsection{Probit regression}\label{probit-regression}}

For probit model calculations are as follow \[
\begin{align*}
    \begin{split}
        \ell^{*}(\btheta) & = \sum_{i \in S_{A}}\log\left\{\frac{\Phi(\bx_{i}^{\top}\btheta)}{1 - \Phi(\bx_{i}^{\top}\btheta)}\right\} + \sum_{i \in S_{B}}d_{i}^{B}\log\{1 - \Phi(\bx_{i}^{\top}\btheta)\}
    \end{split}
\end{align*}
\] with analytical gradient as \[
        \frac{\partial \ell^*}{\partial\btheta} = \sum_{i \in S_A}\frac{\phi(\bx_i^{\top}\btheta)}{\Phi(\bx_i^{\top}\btheta)(1 - \Phi(\bx_i^{\top}\btheta))}\bx_i - \sum_{i \in S_B}d_i^B\frac{\phi(\bx_i^{\top}\btheta)}{1 - \Phi(\bx_i^{\top}\btheta)}\bx_i.
\] \(\hat{\btheta}\) can be found by using the following
Netwon-Raphson's iterative method: \[
\btheta^{(m)} = \btheta^{(m-1)} - \{H(\btheta^{(m-1)}\}^{-1}U(\btheta^{(m-1})),
\] where \(\operatorname{H}\) - hessian, \(\operatorname{U}\) -
gradient.

\hypertarget{general-estimating-equations}{%
\section{General estimating
equations}\label{general-estimating-equations}}

The pseudo score equations derived from Maximum Likelihood Estimation
methods may be replaced by a system of general estimating equations. Let
\(\operatorname{h}\left(\bx\right)\) be the smooth function and \[
\begin{equation}
\mathbf{U}(\btheta)=\sum_{i \in S_A} \mathbf{h}\left(\mathbf{x}_i, \btheta\right)-\sum_{i \in S_B} d_i^B \pi\left(\mathbf{x}_i, \btheta\right) \mathbf{h}\left(\mathbf{x}_i, \btheta\right).
\end{equation}
\] Under \(\operatorname{h}\left(\bx_i\right) = \bx_i\) and logistic
model for propensity score, Equation (2.1) looks like disorted version
of the score equation from MLE method. Then \[
    \mathbf{U}(\btheta)=\sum_{i \in S_A} \bx_i -\sum_{i \in S_B} d_i^B \pi\left(\mathbf{x}_i, \btheta\right) \bx_i.
\] and analytical Jacobian is given by \[
\frac{\partial \mathbf{U}}{\partial\btheta} = - \sum_{i \in S_B} d_i^B \pi_i^A\left(\bx_i^{\mathrm{T}} \btheta \right) \left(1 -  \pi_i^A\left(\bx_i^{\mathrm{T}} \btheta \right)\right)\bx_i \bx_i^{\mathrm{T}}.
\] The second proposed of the smooth function in the literature is
\(\operatorname{h}\left(\bx_i\right) = \bx_i \pi_i^A\left(\bx_i^{\mathrm{T}} \btheta \right)^{-1}\),
for which the \(\operatorname{U}\)-function takes the following form \[
    \mathbf{U}(\btheta)=\sum_{i \in S_A}  \bx_i \pi_i^A\left(\bx_i^{\mathrm{T}} \btheta \right)^{-1} -\sum_{i \in S_B} d_i^B \bx_i.
\] Generally, the goal is to find solution for following system of
equations \[
\begin{equation*}
    \sum_{i \in S_A} \mathbf{h}\left(\mathbf{x}_i, \btheta\right) = \sum_{i \in S_B} d_i^B \pi\left(\mathbf{x}_i, \btheta\right) \mathbf{h}\left(\mathbf{x}_i, \btheta\right)
\end{equation*}
\] In total, we have six models for this estimation method depending on
the \(\operatorname{h}\)-function and the way propensity score is
parameterized. Let us present all of them.

\hypertarget{logistic-regression-1}{%
\subsection{Logistic regression}\label{logistic-regression-1}}

As the one model for logistic regression is presented above, we have
equation under
\(\operatorname{h}\left(\bx_i\right) = \bx_i \pi_i^A\left(\bx_i^{\mathrm{T}} \btheta \right)^{-1}\)
to consider. Analytical jacobian is given by \[
    \frac{\partial \operatorname{U}(\btheta)}{\partial \btheta} = -\sum_{i \in S_A} \frac{1 - \pi_i^A\left(\bx_i^{\mathrm{T}} \btheta \right)}{\pi_i^A\left(\bx_i^{\mathrm{T}} \btheta \right)} \bx_i \bx_i^{\mathrm{T}}.
\]

\hypertarget{complementary-log-log-regression-1}{%
\subsection{Complementary log-log
regression}\label{complementary-log-log-regression-1}}

For
\(\operatorname{h}\left(\bx_i\right) = \bx_i \pi_i^A\left(\bx_i^{\mathrm{T}} \btheta \right)^{-1}\)
analytical jacobian is given by \[
    \frac{\partial \operatorname{U}(\btheta)}{\partial \btheta} = - \sum_{i \in S_A} \frac{1 - \pi_i^A\left(\bx_i^{\mathrm{T}} \btheta \right)}{\pi_i^A\left(\bx_i^{\mathrm{T}} \btheta \right)^2} \exp(\bx_i^{\mathrm{T}} \btheta) \bx_i \bx_i^{\mathrm{T}}
\] and \(\operatorname{h}\left(\bx_i\right) = \bx_i\) we have \[
    \frac{\partial \operatorname{U}(\btheta)}{\partial \btheta} = - \sum_{i \in S_B} \frac{1 - \pi_i^A\left(\bx_i^{\mathrm{T}} \btheta \right)}{\pi_i^B} \exp \left(\bx_i^\mathrm{T} \btheta\right) \bx_i \bx_i^{\mathrm{T}}.
\]

\hypertarget{probit-regression-1}{%
\subsection{Probit regression}\label{probit-regression-1}}

Similarly, for the probit model, under
\(\operatorname{h}\left(\bx_i\right) = \bx_i \pi_i^A\left(\bx_i^{\mathrm{T}} \btheta \right)^{-1}\)
analyical jacobian is given by \[
    \frac{\partial \operatorname{U}(\btheta)}{\partial \btheta} = - \sum_{i \in S_A} \frac{\dot{\pi}_i^A\left(\bx_i^{\mathrm{T}} \btheta \right)}{\pi_i^A\left(\bx_i^{\mathrm{T}} \btheta \right)^2} \bx_i \bx_i^{\mathrm{T}}
\] and under \(\operatorname{h}\left(\bx_i\right) = \bx_i\) we have

\[
    \frac{\partial \operatorname{U}(\partial \btheta)}{\btheta} = - \sum_{i \in S_B} \frac{\dot{\pi}_i^A\left(\bx_i^{\mathrm{T}} \btheta \right)}{\pi_i^B} \bx_i \bx_i^{\mathrm{T}}.
\]

\hypertarget{population-mean-estimator-and-its-properties}{%
\section{Population mean estimator and its
properties}\label{population-mean-estimator-and-its-properties}}

\[
\begin{equation*}
    \hat{\mu}_{IPW1} = \frac{1}{N} \sum_{i \in S_A} \frac{y_i}{\hat{\pi}_i^{A}}
\end{equation*}
\]

\[
\begin{equation*}
    \hat{\mu}_{IPW2} = \frac{1}{\hat{N}^{A}} \sum_{i \in S_A} \frac{y_i}{\hat{\pi}_i^{A}},
\end{equation*}
\] where \(\hat{N^A} = \sum_{i \in S_A} \hat{d}_i^A\).

\hypertarget{variance-of-an-estimator}{%
\subsection{Variance of an estimator}\label{variance-of-an-estimator}}

Let \(\boldsymbol{\eta} = \left(\mu, \btheta^{T}\right)^{T}\) be the set
of parameters to estimate for inverse probability weighting model. The
estimator
\(\hat{\boldsymbol{\eta}} = \left(\hat{\mu}, \hat{\btheta}^{T}\right)^{T}\)
is the solution of joint estimating equations
\(\boldsymbol{\Phi}_n(\boldsymbol{\eta}) = \bZero\). \[
\begin{equation}
\boldsymbol{\Phi}_n(\boldsymbol{\eta})=\left(\begin{array}{c}
\frac{1}{N} \sum_{i=1}^N\left[\frac{R_i\left(y_i-\mu\right)}{\pi_i^A}+\Delta \frac{R_i-\pi_i^A}{\pi_i^A}\right] \\
\mathbf{U}(\btheta)
\end{array}\right)=\bZero,
\end{equation}
\] where where \(\Delta = \mu\), if \(\hat{\mu} = \hat{\mu}_{IPW1}\) and
\(\Delta = 0\) if \(\hat{\mu} = \hat{\mu}_{IPW2}\).
\(\mathbf{U}(\btheta)\) is objective function corresponding to given way
od estimation. For example in case od pseudo maximum likelihood
estimation, we have gradient corresponding to the choosen link function.
In case of GEE it will be on of considered estimating equations. We have
\(\mathbb{E} \left\{\boldsymbol{\Phi}_n(\boldsymbol{\eta})\right\} = \bZero\)
when
\(\boldsymbol{\eta} = \boldsymbol{\eta}_{0} = \left(\mu_y, \btheta_0^{T}\right)^{T}\).
By applying the first order Taylor expansion around
\(\boldsymbol{\eta}_0\), we further have \[
\begin{equation*}
\hat{\boldsymbol{\eta}}-\boldsymbol{\eta}_0=\left[\boldsymbol{\phi}_n\left(\boldsymbol{\eta}_0\right)\right]^{-1} \boldsymbol{\Phi}_n\left(\boldsymbol{\eta}_0\right)+o_p\left(n_A^{-1 / 2}\right)=\left[E\left\{\boldsymbol{\phi}_n\left(\boldsymbol{\eta}_0\right)\right\}\right]^{-1} \boldsymbol{\Phi}_n\left(\boldsymbol{\eta}_0\right)+o_p\left(n_A^{-1 / 2}\right),
\end{equation*}
\] where
\(\boldsymbol{\phi}_n(\boldsymbol{\eta})=\partial \boldsymbol{\Phi}_n(\boldsymbol{\eta}) / \partial \boldsymbol{\eta}\).
It follows that \[
\begin{equation}
\operatorname{Var}(\hat{\boldsymbol{\eta}})=\left[E\left\{\boldsymbol{\phi}_n\left(\boldsymbol{\eta}_0\right)\right\}\right]^{-1} \operatorname{Var}\left\{\boldsymbol{\Phi}_n\left(\boldsymbol{\eta}_0\right)\right\}\left[E\left\{\boldsymbol{\phi}_n\left(\boldsymbol{\eta}_0\right)\right\}^{\top}\right]^{-1}+o\left(n_A^{-1}\right).
\end{equation}
\] Let us show how to derive this variance-covariance matrix in case of
MLE with logistic regression model. Calculations for the rest of the
models is available in Appendix and you are welcome to take a look at
them. Thus, we have \[
\boldsymbol{\Phi}_n(\boldsymbol{\eta})=\left(\begin{array}{c}
\frac{1}{N} \sum_{i=1}^N\left[\frac{R_i\left(y_i-\mu\right)}{\pi_i^A}+\Delta \frac{R_i-\pi_i^A}{\pi_i^A}\right] \\
\frac{1}{N} \sum_{i=1}^N R_i \boldsymbol{x}_i-\frac{1}{N} \sum_{i \in \mathcal{S}_B} d_i^B \pi_i^A \boldsymbol{x}_i
\end{array}\right)
\] and

\[
\phi_n(\boldsymbol{\eta})=\frac{1}{N}\left(\begin{array}{cc}
-\sum_{i=1}^N\left\{\left(1-\frac{\Delta}{\mu}\right) \frac{R_i}{\pi_i^A}+\frac{\Delta}{\mu}\right\} & -\sum_{i=1}^N \frac{1-\pi_i^A}{\pi_i^A} R_i\left(y_i-\mu+\Delta\right) \boldsymbol{x}_i^{\top} \\
\mathbf{0} & -\sum_{i \in \mathcal{S}_B} d_i^B \pi_i^A\left(1-\pi_i^A\right) \boldsymbol{x}_i \boldsymbol{x}_i^{\top}
\end{array}\right)
\] It can be shown that \[
\left[E\left\{\boldsymbol{\phi}_n(\boldsymbol{\eta})\right\}\right]^{-1}=\left(\begin{array}{cc}
-1 &  \mathbf{b}^{\top} \\
\mathbf{0} & -\left[\frac{1}{N} \sum_{i=1}^N \pi_i^A\left(1-\pi_i^A\right) \boldsymbol{x}_i \boldsymbol{x}_i^{\top}\right]^{-1}
\end{array}\right)
\] where

\[
\mathbf{b}^{\top} = 
\left\{N^{-1} \sum_{i=1}^N\left(1-\pi_i^{\mathrm{A}}\right) \left(y_i-\mu_y + \Delta\right) x_i^{\top}\right\}\left\{N^{-1} \sum_{i=1}^N \pi_i^{\mathrm{A}}\left(1-\pi_i^{\mathrm{A}}\right) \boldsymbol{x}_i x_i^{\top}\right\}^{-1}
\]

\(\operatorname{Var}\left\{\boldsymbol{\Phi}_n\left(\boldsymbol{\eta}_0\right)\right\}\)
can be found using decomposition of
\(\bPhi_n(\boldsymbol{\eta}) = \bA_1 + \bA_2\). Then
\(\operatorname{Var}\left\{\boldsymbol{\Phi}_n\left(\boldsymbol{\eta}_0\right)\right\} = \operatorname{Var}\left(\bA_1\right) + \operatorname{Var}\left(\bA_2\right)\).
Let \[
\mathbf{A}_1=\frac{1}{N} \sum_{i=1}^N\left(\begin{array}{c}
\frac{R_i\left(y_i-\mu\right)}{\pi_i^A}+\Delta \frac{R_i-\pi_i^A}{\pi_i^A} \\
R_i \boldsymbol{x}_i-\pi_i^A \boldsymbol{x}_i
\end{array}\right), \quad \mathbf{A}_2=\frac{1}{N}\left(\begin{array}{c}
0 \\
\sum_{i=1}^N \pi_i^A \boldsymbol{x}_i-\sum_{i \in S_B} d_i^B \pi_i^A \boldsymbol{x}_i
\end{array}\right)
\]

With this division, we have
\(\operatorname{Var}\left\{\boldsymbol{\Phi}_n\left(\boldsymbol{\eta}_0\right)\right\}=\mathbf{V}_1+\mathbf{V}_2\)
where \(\mathbf{V}_1 = Var \left(A_1\right)\) and
\(\mathbf{V}_2 = Var \left(A_2\right)\). \(V_1\) depends only on the
model for propensity score and \(V_2\) on sampling design for
probability sample, both evaluated on
\(\boldsymbol{\eta} = \boldsymbol{\eta}_0\). Finally we have \[
\mathbf{V}_1=\frac{1}{N^2} \sum_{i=1}^N\left(\begin{array}{cc}
\left\{\left(1-\pi_i^A\right) / \pi_i^A\right\}\left(y_i-\mu+\Delta\right)^2 & \left(1-\pi_i^A\right)\left(y_i-\mu+\Delta\right) \boldsymbol{x}_i^{\top} \\
\left(1-\pi_i^A\right)\left(y_i-\mu+\Delta\right) \boldsymbol{x}_i & \pi_i^A\left(1-\pi_i^A\right) \boldsymbol{x}_i \boldsymbol{x}_i^{\top}
\end{array}\right)
\] and \[
\mathbf{V}_2=\left(\begin{array}{ll}
0 & \mathbf{0}^{\top} \\
\mathbf{0} & \mathbf{D}
\end{array}\right)
\] where
\(\mathbf{D}=N^{-2} V_p\left(\sum_{i \in \mathcal{S}_B} d_i^B \pi_i^A \boldsymbol{x}_i\right)\)
and is given by \[
\begin{equation}
{\mathbf{D}}=\frac{1}{N^2} \sum_{i \in \mathcal{S}_{\mathrm{B}}} \sum_{j \in \mathcal{S}_{\mathrm{B}}} \frac{\pi_{i j}^{\mathrm{B}}-\pi_i^{\mathrm{B}} \pi_j^{\mathrm{B}}}{\pi_{i j}^{\mathrm{B}}} \frac{{\pi}_i^{\mathrm{A}}}{\pi_i^{\mathrm{B}}} \frac{{\pi}_j^{\mathrm{A}}}{\pi_j^{\mathrm{B}}} \boldsymbol{x}_i \boldsymbol{x}_j^{\top}
\end{equation}
\] The asymptotic variance for \(\mu_{IPW}\) is the first diagonal
element of matrix \[
\left[E\left\{\boldsymbol{\phi}_n\left(\boldsymbol{\eta}_0\right)\right\}\right]^{-1} \operatorname{Var}\left\{\boldsymbol{\Phi}_n\left(\boldsymbol{\eta}_0\right)\right\}\left[E\left\{\boldsymbol{\phi}_n\left(\boldsymbol{\eta}_0\right)\right\}^{\top}\right]^{-1}.
\]

\bookmarksetup{startatroot}

\hypertarget{mass-imputation}{%
\chapter{Mass imputation}\label{mass-imputation}}

\hypertarget{motivation-and-assumptions-1}{%
\section{Motivation and
assumptions}\label{motivation-and-assumptions-1}}

\hypertarget{model-based-methods}{%
\section{Model-based methods}\label{model-based-methods}}

This method is based on parametric model on sample \(S_A\) in the form
of

\[
\begin{equation}
\mathbb{E}\left(y_i \mid \bx_i\right) = m\left(\bx_i, \bbeta_{0}\right)
\end{equation}
\]

for some unknown \(\bbeta_{0}\) and known function \(m(\cdot)\). The
specification of m-function typically follows the mean function for
generalized linear models. If \(\by\) is continous, we can use linear
regression model with
\(m\left(\boldsymbol{x}_i, \bbeta_{0}\right) = \bx_i^{T} \bbeta_0\). If
\(\by\) is binary, one can use logistic regression model and let
\(m\left(\boldsymbol{x}_i, \bbeta_{0}\right) = \frac{\exp\left( \bx_i^{T} \bbeta_0\right)}{\exp\left( \bx_i^{T} \bbeta_0\right) + 1}\).
If \(\by\) represents count data, we can use log-linear model, where
\(m\left(\boldsymbol{x}_i, \bbeta_{0}\right) = \exp\left(\bx_i^{T} \bbeta_0\right)\)

Mass imputation estimator

\[
\begin{equation}
\frac{1}{\hat{N}^{\mathrm{B}}} \sum_{i \in \mathcal{S}_{\mathrm{B}}} d_i^{\mathrm{B}} m\left(\boldsymbol{x}_i, \hat{\bbeta}\right)
\end{equation}
\] Variance of an estimator

\[
\text{var}\left(\hat{\mu}_{MI}\right) = \text{var}_A + \text{var}_B 
\] and \(\text{var}_A\) can be estimated by \[
\text{var}_A = \frac{1}{n_A} \sum_{i \in S_A} \hat{e}_i^2 \left(\bx_i^{T} \hat{c}\right)^2,
\] where
\(\hat{e_i} = y_i - m\left(\boldsymbol{x}_i, \hat{\bbeta}\right)\) and
\(\hat{c} = \left\{\frac{1}{n_A} \sum_{i \in S_A} \dot{m}\left(\boldsymbol{x}_i, \hat{\bbeta}\right) \bx_i^T \right\}^{-1} N^{-1} \sum_{i \in S_B} d_i^B \bx_i\).
Respectively \(\text{var}_B\) can be estimated by \[
\hat{\text{var}}_B = \frac{1}{N^2} \sum_{i \in \mathcal{S}_{\mathrm{B}}} \sum_{j \in \mathcal{S}_{\mathrm{B}}} \frac{\pi_{i j}^{\mathrm{B}}-\pi_i^{\mathrm{B}} \pi_j^{\mathrm{B}}}{\pi_{i j}^{\mathrm{B}}} d_i^B m\left(\boldsymbol{x}_i, \hat{\bbeta}\right) d_j^B m\left(\boldsymbol{x}_j, \hat{\bbeta}\right)
\]

\hypertarget{nearest-neighbor-imputation}{%
\section{Nearest neighbor
imputation}\label{nearest-neighbor-imputation}}

\hypertarget{assumptions}{%
\subsection{Assumptions}\label{assumptions}}

\hypertarget{model}{%
\subsection{model}\label{model}}

On the other hand we can applied non-parametric method to this problem,
such as nearest neighbor algorithm, that is, find the closest matching
unit from sample \(S_B\) based on the \(\bx\) values and use the
corresponding \(\by\) value from this unit as the imputed value.
Procedure contains two steps

\begin{enumerate}
\def\labelenumi{\arabic{enumi}.}
\item
  for each \(i \in S_B\) find the nearest neighbor from sample \(S_A\).
\item
  Calculate the nearest neighbor imputation estimator of \(\mu_y\) \[
  \begin{equation}
  \hat{\mu}_\mathrm{nn}=\frac{1}{N} \sum_{i \in S_B} d_i^B y_{i(1)} .
  \end{equation}
  \]
\end{enumerate}

Variance of an estimator

We have \[
V_{\mathrm{nni}}=\lim _{n \rightarrow \infty} \frac{n}{N^2} E\left[\operatorname{var}_p\left\{\sum_{i \in S_B} d_i^B g\left(y_i\right)\right\}\right] .
\]

which can be estimated by \[
\hat{\text{var}}_{\mathrm{nni}}=\frac{n}{N^2} \sum_{i \in S_A} \sum_{j \in S_A} \frac{\pi_{i j}-\pi_i \pi_j}{\pi_i \pi_j} \frac{y_{i(1)}}{\pi_i} \frac{y_{j(1)}}{\pi_j}
\]

\hypertarget{k-nearest-neighbor-imputation}{%
\section{K-nearest neighbor
imputation}\label{k-nearest-neighbor-imputation}}

Steps

\begin{enumerate}
\def\labelenumi{\arabic{enumi}.}
\tightlist
\item
  For each unit \(i \in S_B\) find k-nearest neighbors from sample
  \(S_A\). Impute the \(\by\) value for unit \(i\) by
  \(\hat{\mu}\left(\mathbf{x}_i\right)=k^{-1} \sum_{j=1}^k y_{i(j)}\).
\item
  Calculate k-nearest neighbor imputation estimator of \(\mu_y\) \[
  \hat{\mu}_{\mathrm{knn}}=\frac{1}{N} \sum_{i \in S_B} d_i^B \hat{\mu}\left(\mathbf{x}_i\right) .
  \]
\end{enumerate}

Variance of an estimator

We have

\[
\text{var}_{\mathrm{knn}}=\lim _{n \rightarrow \infty} \frac{n}{N^2}\left(E\left[\operatorname{var}_p\left\{\sum_{i \in S_B} d_i^B \mu\left(\bx_i\right)\right\}\right]+E\left\{\frac{1-\pi_A(\bx)}{\pi_A(\bx)} \sigma^2(\bx)\right\}\right),
\] where \(\sigma^2(\bx)=\operatorname{var}\{y \mid \bx\}\) and
\(\pi_A(\bx) = P\left(R_i=1 \mid \boldsymbol{x}\right)\)

\bookmarksetup{startatroot}

\hypertarget{doubly-robust-methods}{%
\chapter{Doubly robust methods}\label{doubly-robust-methods}}

\hypertarget{bias-minimization-technique}{%
\section{Bias minimization
technique}\label{bias-minimization-technique}}

This model is derived from the form of the bias of doubly robust
estimator. We consider set of estimating equations, for which we want to
find regression parameters (\(\btheta, \bbeta\)). Shu Yang, Jae Kwang
Kim and Rui Song proposed this method with logistic regression for
selection model. As before our goal is to expand this approach. At the
beginning let us derive bias of the estimator. We have

\[
\begin{aligned}
bias(\hat{\mu}_{DR}) = & \mathbb{E}\left\{\hat{\mu}_{DR} - \mu\right\} \\ = & \mathbb{E}\left[ \frac{1}{N} \sum_{i=1}^N \left\{\frac{R_i^A}{\pi_i^A \left(\bx_i^{\mathrm{T}} \btheta \right)}  - 1\right\} \left\{y_i - \operatorname{m}\left( \bx_i^{\mathrm{T}} \bbeta\right)\right\} \right] \\ + & \mathbb{E}\left[  \frac{1}{N} \sum_{i=1}^N \left(R_i^B d_i^B - 1\right) \operatorname{m}\left( \bx_i^{\mathrm{T}} \bbeta \right)\right]
\end{aligned}
\] Since we actually care about minimizing the square of the bias, let's
calculate its derivative against the parameter vector. \[
\begin{aligned}
    \frac{\partial \operatorname{bias}(\hat{\mu}_{DR})^2}{\partial \left(\bbeta^{\mathrm{T}}, \btheta^{\mathrm{T}}\right)^{\mathrm{T}}} = 2 \operatorname{bias}(\hat{\mu}_{DR}) J(\theta, \beta),
\end{aligned}
\] where \(J(\theta, \beta)\) is internal derivative and depends on the
model for outcome variable and propensity score. In the basic setting
with propensity score modelling by logistic regression we have following
system of equations to solve \[
\begin{equation}
\begin{aligned}
J(\theta, \beta)=\left(\begin{array}{c}
J_1(\theta, \beta) \\
J_2(\theta, \beta)
\end{array}\right)=\left(\begin{array}{c}
\sum_{i=1}^N R_i^A\left\{\frac{1}{\pi\left(\boldsymbol{x}_i, \boldsymbol{\theta}\right)}-1\right\}\left\{y_i-m\left(\boldsymbol{x}_i, \boldsymbol{\beta}\right)\right\} \boldsymbol{x}_i \\
\sum_{i=1}^N \frac{R_i^A}{\pi\left(\boldsymbol{x}_i, \boldsymbol{\theta}\right)} \frac{\partial m\left(\boldsymbol{x}_i, \boldsymbol{\beta}\right)}{\partial \bbeta}  - \sum_{i \in \mathcal{S}_{\mathrm{B}}} d_i^{\mathrm{B}} \frac{\partial m\left(\boldsymbol{x}_i, \boldsymbol{\beta}\right)}{\partial \bbeta} 
\end{array}\right)
\end{aligned}
\end{equation}
\] where \(\left(\boldsymbol{x}_i, \boldsymbol{\beta}\right)\) is
working model for outcome variable, for example in linear regression
case we have \[
m\left(\boldsymbol{x}_i, \boldsymbol{\beta}\right) = \bx_i^{T} \bbeta
\] and \[
\frac{\partial m\left(\boldsymbol{x}_i, \boldsymbol{\beta}\right)}{\partial \bbeta} = \bx_i.
\] For complementary log-log model we have \[
\begin{equation}
J(\theta, \beta)=\left(\begin{array}{c}
J_1(\theta, \beta) \\
J_2(\theta, \beta)
\end{array}\right)=\left(\begin{array}{c}
\frac{1}{N} \sum_{i=1}^N R_i^A\left\{\frac{1 - \pi_i^A\left(\bx_i^{\mathrm{T}} \btheta \right)}{\pi_i^A\left(\bx_i^{\mathrm{T}} \btheta \right)^2} \exp(\bx_i^{\mathrm{T}} \btheta)\right\}\left\{y_i-m\left(\bx_i^{\mathrm{T}} \beta\right)\right\} \bx_i \\
\frac{1}{N} \sum_{i=1}^N\left\{\frac{R_i^A}{\pi_i^A\left(\bx^{\mathrm{T}}\theta\right)}-d_i^B R_i^B\right\} \frac{\partial m\left(\bx_i^{\mathrm{T}} \beta\right)}{\partial \beta}
\end{array}\right)
\end{equation}
\] and probit model

\[
\begin{equation}
J(\theta, \beta)=\left(\begin{array}{c}
J_1(\theta, \beta) \\
J_2(\theta, \beta)
\end{array}\right)=\left(\begin{array}{c}
\frac{1}{N} \sum_{i=1}^N R_i^A\frac{\dot{\pi_i^A}\left(\bx_i^{\mathrm{T}} \btheta \right)}{\pi_i^A\left(\bx_i^{\mathrm{T}} \btheta \right)^2} \left\{y_i-m\left(\bx_i^{\mathrm{T}} \beta\right)\right\} \bx_i \\
\frac{1}{N} \sum_{i=1}^N\left\{\frac{R_i^A}{\pi_i^A\left(\bx^{\mathrm{T}}\theta\right)}-d_i^B R_i^B\right\} \frac{\partial m\left(\bx_i^{\mathrm{T}} \beta\right)}{\partial \beta}
\end{array}\right)
\end{equation}
\]

Goal is to solve following system of equations \[
J(\theta, \beta)=\bZero
\]

\hypertarget{population-mean-estimator-and-its-properties-1}{%
\section{Population mean estimator and its
properties}\label{population-mean-estimator-and-its-properties-1}}

\[
\begin{equation*}
    \hat{\mu}_{\mathrm{DR}}=\frac{1}{\hat{N}^{\mathrm{A}}} \sum_{i \in \mathcal{S}_{\mathrm{A}}} d_i^{\mathrm{A}}\left\{y_i-m\left(\boldsymbol{x}_i, \hat{\boldsymbol{\beta}}\right)\right\}+\frac{1}{\hat{N}^{\mathrm{B}}} \sum_{i \in \mathcal{S}_{\mathrm{B}}} d_i^{\mathrm{B}} m\left(\boldsymbol{x}_i, \hat{\boldsymbol{\beta}}\right),
\end{equation*}
\] where \(d_i^A = \pi \left(\bx_i, \btheta\right)^{-1}\),
\(\hat{N}^A = \sum_{i \in S_A} d_i^A\) and
\(\hat{N}^B = \sum_{i \in S_B} d_i^B\). We will first show how to obtain
the variance of the derived estimator using separate procedures for
propensity score and mass imputation (for example MLE and linear
regression), and then variance derived from bias minimization technique
for estimation.

It can be shown that parameters \(\bbeta\) have no impact on asymptotic
variance of \(\hat{\mu}_{DR}\). Let's assume that
\(\hat{\bbeta} - \bbeta^* = O_p(n_A^{-1/2})\) for some fixed
\(\bbeta^*\). Notice that the first part of DR estimator is the
\(\mu_{IPW2}\) estimator with \(y_i\) replaced by
\(y_i - \operatorname{m}\left(\bx_i,\bbeta^*\right)\). Using the
asymptotic expansions developed for \(\mu_{IPW2}\) and logistic
regression model for propensity score we have \[
\begin{equation}
\begin{aligned}
\frac{1}{\hat{N}^A} \sum_{i=1}^N \frac{R_i\left\{y_i-m\left(\boldsymbol{x}_i, \boldsymbol{\beta}^*\right)\right\}}{\hat{\pi}_i^A}= & h_N+\frac{1}{N} \sum_{i=1}^N R_i\left\{\frac{y_i-m\left(\boldsymbol{x}_i, \boldsymbol{\beta}^*\right)-h_N}{\pi_i^A}-\mathbf{b}_3^{\top} \boldsymbol{x}_i\right\} \\
& +\mathbf{b}^{\top} \frac{1}{N} \sum_{i \in \mathcal{S}_B} d_i^B \pi_i^A \boldsymbol{x}_i+o_p\left(n_A^{-1 / 2}\right),
\end{aligned}
\end{equation}
\] where
\(h_N=N^{-1} \sum_{i=1}^N\left\{y_i-m\left(\boldsymbol{x}_i, \boldsymbol{\beta}^*\right)\right\}\)
and

\[
\mathbf{b}^{\top}=\left[\frac{1}{N} \sum_{i=1}^N\left(1-\pi_i^A\right)\left\{y_i-m\left(\boldsymbol{x}_i, \boldsymbol{\beta}^*\right)-h_N\right\} \boldsymbol{x}_i^{\top}\right]\left\{\frac{1}{N} \sum_{i=1}^N \pi_i^A\left(1-\pi_i^A\right) \boldsymbol{x}_i \boldsymbol{x}_i^{\top}\right\}^{-1}
\] The second part of the estimator has the following expansion \[
\frac{1}{\hat{N}^B} \sum_{i \in \mathcal{S}_B} d_i^B m\left(\bx_i,\bbeta^*\right)=\frac{1}{N} \sum_{i=1}^N m\left(\bx_i,\bbeta^*\right)+\frac{1}{N} \sum_{i \in \mathcal{S}_B} d_i^B\left\{m\left(\bx_i,\bbeta^*\right)-\frac{1}{N} \sum_{i=1}^N m_i\right\}+O_p\left(n_B^{-1}\right)
\] Further, putting these two parts together, we have \[
\hat{\mu}_{D R 2}-\mu_y=\frac{1}{N} \sum_{i=1}^N R_i\left\{\frac{y_i-m\left(\boldsymbol{x}_i, \boldsymbol{\beta}^*\right)-h_N}{\pi_i^A}-\mathbf{b}_3^{\top} \boldsymbol{x}_i\right\}+\frac{1}{N} \sum_{i \in \mathcal{S}_B} d_i^B t_i+o_p\left(n_A^{-1 / 2}\right),
\] where
\(t_i=\pi_i^A \boldsymbol{x}_i^{\top} \mathbf{b}_3+m\left(\boldsymbol{x}_i, \boldsymbol{\beta}^*\right)-N^{-1} \sum_{i=1}^N m\left(\boldsymbol{x}_i, \boldsymbol{\beta}^*\right)\).
Finally \[
\begin{gathered} \text{var}(\hat{\mu}_{DR}) =
\frac{1}{N^2} \sum_{i=1}^N\left(1-\pi_i^{\mathrm{A}}\right) \pi_i^{\mathrm{A}}\left[\left\{y_i-m\left(\boldsymbol{x}_i, \boldsymbol{\beta}^*\right)-h_{\mathrm{N}}\right\} /\right. \\
\left.\pi_i^{\mathrm{A}}-\mathbf{b}_3^{\top} \boldsymbol{x}_i\right]^2+W,
\end{gathered}
\] where
\(W = N^{-2} \text{var}_p\left(\sum_{i \in \mathcal{S}_{\mathrm{B}}} d_i^{\mathrm{B}} t_i\right)\).

When model is obtained by using systems of equations derived from bias
minimization technique. We have variance decomposition on probability
and nonprobability part, where \[
\text{var}_B = \mathbb{E} \left\{\frac{1}{N^2} \sum_{i \in \mathcal{S}_{\mathrm{B}}} \sum_{j \in \mathcal{S}_{\mathrm{B}}} \frac{\pi_{i j}^{\mathrm{B}}-\pi_i^{\mathrm{B}} \pi_j^{\mathrm{B}}}{\pi_{i j}^{\mathrm{B}}} d_i^B m\left(\boldsymbol{x}_i, \bbeta^*\right) d_j^B m\left(\boldsymbol{x}_j, \bbeta^*\right) \right\}
\] which can be estimated by \[
\hat{\text{var}}_B = \frac{1}{N^2} \sum_{i \in \mathcal{S}_{\mathrm{B}}} \sum_{j \in \mathcal{S}_{\mathrm{B}}} \frac{\pi_{i j}^{\mathrm{B}}-\pi_i^{\mathrm{B}} \pi_j^{\mathrm{B}}}{\pi_{i j}^{\mathrm{B}}} d_i^B m\left(\boldsymbol{x}_i, \hat{\bbeta}\right) d_j^B m\left(\boldsymbol{x}_j, \hat{\bbeta}\right)
\] For non-probability part we have \[
\text{var}_A = \frac{1}{N^2} \sum_{i=1}^N \mathbb{E} \left[R_i^A \left\{ \frac{1 - 2 \pi_i^A}{\left( \pi_i^A \right) ^2} \right\} \sigma_i^2 + \sigma_i^2 \right]
\] what can be estimated by \[
\hat{\text{var}}_A = \frac{1}{N^2} \sum_{i=1}^N R_i^A \left\{ \frac{1 - 2 \hat{\pi}_i^A}{\left( \hat{\pi}_i^A \right) ^2} \right\} \hat{\sigma}_i^2 + \sum_{i \in S_B} d_i^B \hat{\sigma}_i^2,
\] where \(\hat{\sigma}_i^2\) is consistent estimator of
\(\sigma_i^2 = \text{var}\left(y_i \mid \bx_i\right)\).

\bookmarksetup{startatroot}

\hypertarget{techniques-of-variables-selection-for-high-dimensional-data}{%
\chapter{Techniques of variables selection for high-dimensional
data}\label{techniques-of-variables-selection-for-high-dimensional-data}}

Let \(\operatorname{U}\left(\btheta, \bbeta\right)\) be the join
estimating function for \(\left(\btheta, \bbeta\right)\). When p is
large, we consider the penalized estimating functions for
\(\left(\btheta, \bbeta\right)\) as

\[
\operatorname{U}^p\left(\btheta, \bbeta\right) = \operatorname{U}\left(\btheta, \bbeta\right) -\left(\begin{array}{c}
q_{\lambda_\theta}(|\btheta|) \operatorname{sgn}(\btheta) \\
q_{\lambda_\beta}(|\bbeta|) \operatorname{sgn}(\bbeta)
\end{array}\right),
\] where \(q_{\lambda_{\theta}}\) and \(q_{\lambda_{\beta}}\) are some
smooth functions. We let
\(q_{\lambda}\left(x\right) = \frac{\partial p_{\lambda}}{\partial x}\),
where \(p_{\lambda}\) is some penalization function.

\hypertarget{lasso}{%
\section{LASSO}\label{lasso}}

\[
p_{\lambda}(x) = \lambda |x|
\] and its derivative

\[
p_{\lambda}(x)= \begin{cases} - \lambda & \text { if }x < 0, \\ \left[-\lambda, \lambda\right] & \text { if } x = 0 \\ \lambda & \text { if }x > 0\end{cases}
\]

\hypertarget{scad}{%
\section{SCAD}\label{scad}}

\[
p_{\lambda}(x ; \gamma)= \begin{cases}\lambda|x| & \text { if }|x| \leq \lambda, \\ \frac{2 \gamma \lambda|x|-x^2-\lambda^2}{2(\gamma-1)} & \text { if } \lambda<|x|<\gamma \lambda \\ \frac{\lambda^2(\gamma+1)}{2} & \text { if }|x| \geq \gamma \lambda\end{cases}
\] and derivative is \[
q_{\lambda}(x ;  \gamma)= \begin{cases}\lambda & \text { if }|x| \leq \lambda \\ \frac{\gamma \lambda-|x|}{\gamma-1} & \text { if } \lambda<|x|<\gamma \lambda \\ 0 & \text { if }|x| \geq \gamma \lambda\end{cases}
\]

\hypertarget{mcp}{%
\section{MCP}\label{mcp}}

\[
p_{\lambda}(x ; \gamma)= \begin{cases}\lambda|x|-\frac{x^2}{2 \gamma}, & \text { if }|x| \leq \gamma \lambda \\ \frac{1}{2} \gamma \lambda^2, & \text { if }|x|>\gamma \lambda\end{cases}
\] and derivative is \[
q_\lambda(x ; \gamma)= \begin{cases}\left(\lambda-\frac{|x|}{\gamma}\right) \operatorname{sign}(x), & \text { if }|x| \leq \gamma \lambda, \\ 0, & \text { if }|x|>\gamma \lambda\end{cases}
\]

\hypertarget{solution}{%
\section{Solution}\label{solution}}

By minorization-maximization algorithm, the penalized estimator
\(\left(\hat{\btheta}, \hat{\bbeta}\right)\) satisfies

\[
\operatorname{U}^p\left(\hat{\btheta}, \hat{\bbeta}\right) = \operatorname{U}\left(\hat{\btheta}, \hat{\bbeta}\right) -\left(\begin{array}{c}
q_{\lambda_\hat{\theta}}(|\hat{\btheta}|) \operatorname{sgn}(\hat{\btheta}) \frac{|\hat{\btheta}|}{\epsilon + |\hat{\btheta}|} \\
q_{\lambda_\hat{\beta}}(|\hat{\bbeta}|) \operatorname{sgn}(\hat{\bbeta}) \frac{|\hat{\bbeta}|}{\epsilon + |\hat{\bbeta}|}
\end{array}\right) = \bZero
\] Let
\(\nabla\left(\btheta, \bbeta \right) = \frac{\partial \operatorname{U}\left(\btheta, \bbeta\right)}{\partial \left(\btheta^{T} \bbeta^{T}\right)^{T}} = Diag \left(\frac{\partial U_1 \left(\btheta \right)}{\partial \btheta^{T}}, \frac{\partial U_2 \left(\bbeta \right)}{\partial \bbeta^{T}} \right)\),
where \(U_1\) is objective function for selection model and \(U_2\) for
outcome model. Let
\(\boldsymbol{\alpha} = \left(\btheta, \bbeta\right)\) and

\[
\Lambda(\boldsymbol{\alpha})=\left(\begin{array}{ccc}
q_{\lambda_1}\left(\left|\alpha_1\right|\right) & \ldots & 0 \\
\vdots & \ddots & \vdots \\
0 & \ldots & q_{\lambda_{2 p}}\left(\left|\alpha_{2 p}\right|\right)
\end{array}\right)
\] Newton-Raphson procedure for j-variable and k-update

\[
\hat{\alpha}_j^{[k]}=\hat{\alpha}_j^{[k-1]}+\left\{\nabla_{j j}\left(\hat{\alpha}^{[k-1]}\right)+N \Lambda_{j j}\left(\hat{\alpha}^{[k-1]}\right)\right\}^{-1}\left\{U_j\left(\hat{\alpha}^{[k-1]}\right)-N \Lambda_{j j}\left(\hat{\alpha}^{[k-1]}\right) \hat{\alpha}_j^{[k-1]}\right\}
\]

It is recommended to use K-fold cross validation for selectiing tuning
parameters \(\left(\lambda_{\theta}, \lambda_{\beta}\right)\) which
minimize following loss functions for set of parameters \(\balpha\).

\[
\operatorname{Loss}\left(\lambda_\theta\right)=\sum_{j=1}^p\left(\sum_{i=1}^N\left[\frac{R_i^A}{\pi_i^A\left\{\bx_i^{\mathrm{T}} \hat{\theta}\left(\lambda_\theta\right)\right\}}-\frac{I_{\mathrm{A}, i}}{\pi_{\mathrm{A}, i}}\right] \bx_{i, j}\right)^2,
\]

\[
\operatorname{Loss}\left(\lambda_\beta\right)=\sum_{i=1}^N R_i^A\left[y_i-m\left\{\bx_i^{\mathrm{T}} \hat{\beta}\left(\lambda_\beta\right)\right\}\right]^2,
\] where \(\hat{\theta}\left(\lambda_\theta\right)\) and
\(\hat{\beta}\left(\lambda_\beta\right)\) are penalized estimators with
tuning parameters \(\lambda_\theta\), \(\lambda_\beta\) for selection
and outcome model respectively. For estimation we consider only the
union of covariates \(\bX_C\), where
\(C = \hat{M}_{\theta} + \hat{M}_{\beta}\) and
\(\hat{M}_{\theta} = \left\{j: \hat{\theta}_j \ne 0\right\}\) and
\(\hat{M}_{\beta} = \left\{j: \hat{\beta}_j \ne 0\right\}\). In short,
we estimate only on truly important variables for selection and outcome
models.

\bookmarksetup{startatroot}

\hypertarget{summary}{%
\chapter{Summary}\label{summary}}

\bookmarksetup{startatroot}

\hypertarget{appendices}{%
\chapter{Appendices}\label{appendices}}

\[
\newcommand{\bSigma}{\boldsymbol{\Sigma}}
\newcommand{\bOmega}{\boldsymbol{\Omega}}
\newcommand{\bTheta}{\boldsymbol{\Theta}}
\newcommand{\bPi}{\boldsymbol{\Pi}}
\newcommand{\bbeta}{\boldsymbol{\beta}}
\newcommand{\balpha}{\boldsymbol{\alpha}}
\newcommand{\brho}{\boldsymbol{\rho}}
\newcommand{\beps}{\boldsymbol{\epsilon}}
\newcommand{\blambda}{\boldsymbol{\lambda}}
\newcommand{\bgamma}{\boldsymbol{\gamma}}
\newcommand{\btheta}{\boldsymbol{\theta}}
\newcommand{\bmu}{\boldsymbol{\mu}}
\newcommand{\bpi}{\boldsymbol{\pi}}
\newcommand{\bphi}{\boldsymbol{\phi}}
\newcommand{\bPhi}{\boldsymbol{\Phi}}
\newcommand{\boldeta}{\boldsymbol{\eta}}
\newcommand{\bx}{\boldsymbol{x}}
\newcommand{\bD}{\boldsymbol{D}}
\newcommand{\bV}{\boldsymbol{V}}
\newcommand{\bv}{\boldsymbol{v}}
\newcommand{\bY}{\boldsymbol{Y}}
\newcommand{\bA}{\boldsymbol{A}}
\newcommand{\bB}{\boldsymbol{B}}
\newcommand{\bR}{\boldsymbol{R}}
\newcommand{\bM}{\boldsymbol{M}}
\newcommand{\bI}{\boldsymbol{I}}
\newcommand{\bC}{\boldsymbol{C}}
\newcommand{\bW}{\boldsymbol{W}}
\newcommand{\bw}{\boldsymbol{w}}
\newcommand{\bd}{\boldsymbol{d}}
\newcommand{\bT}{\boldsymbol{T}}
\newcommand{\bt}{\boldsymbol{t}}
\newcommand{\bZ}{\boldsymbol{Z}}
\newcommand{\bX}{\boldsymbol{X}}
\newcommand{\bz}{\boldsymbol{z}}
\newcommand{\by}{\boldsymbol{y}}
\newcommand{\br}{\boldsymbol{r}}
\newcommand{\bp}{\boldsymbol{p}}
\newcommand{\bb}{\boldsymbol{b}}
\newcommand{\bZero}{\boldsymbol{0}}
\newcommand{\bOne}{\boldsymbol{1}}
\]

\hypertarget{ipw-estimator-variance}{%
\section{IPW estimator variance}\label{ipw-estimator-variance}}

Similarly as in second chapter wi will derive variance of an IPW
estimator, but this time we will consider all of the other models for
propensity score. In the first part we expand Taylor approximation for
MLE model with complementary log-log and probit regression for
propensity score. In the next part we this approach for variance
estimation in GEE models.

\hypertarget{mle-complementary-log-log-model}{%
\subsection{MLE (complementary log-log
model)}\label{mle-complementary-log-log-model}}

Recall that we have following system of equation to solve for cloglog
model with corresponding objective function (gradient).

\[
\begin{equation*}
\boldsymbol{\Phi}_n(\boldsymbol{\eta})=\left(\begin{array}{c}
\frac{1}{N} \sum_{i=1}^N\left[\frac{R_i\left(y_i-\mu\right)}{\pi_i^A}+\Delta \frac{R_i-\pi_i^A}{\pi_i^A}\right] \\
\frac{1}{N}\sum_{i=1}^N R_i \frac{\exp(\bx_{i}^{\top}\btheta)\bx_{i}}{\pi(\bx_{i}, \btheta)} - \sum_{i \in S_{B}}d_{i}^{B}\exp(\bx_{i}^{T}\btheta)\bx_{i}
\end{array}\right) = \bZero
\end{equation*}
\] for which derivative along \(\boldsymbol{\eta}\) is

\[
\begin{equation*}
\phi_n(\boldsymbol{\eta})=\frac{1}{N}\left(\begin{array}{cc}
-\sum_{i=1}^N\left\{\left(1-\frac{\Delta}{\mu}\right) \frac{R_i}{\pi_i^A}+\frac{\Delta}{\mu}\right\} & \sum_{i=1}^N \frac{R_i\left(y_i-\mu + \Delta\right)\left(1 - \pi_i^A\right)\log \left(1 - \pi_i^A\right)}{\left(\pi_i^A\right)^2} \boldsymbol{x}_i^{\top} \\
\mathbf{0} & \operatorname{H}(\btheta)
\end{array}\right),
\end{equation*}
\] where \(\operatorname{H}\) is hessian in point \(\btheta\). We can
show that \[
\begin{equation*}
\left[E(\phi_n(\boldsymbol{\eta}))\right]^{-1}=\left(\begin{array}{cc}
-1 & \mathbf{b}^T \\
\mathbf{0} & \left[\frac{1}{N}\operatorname{E}\left(\operatorname{H}(\btheta)\right)\right]^{-1}
\end{array}\right),
\end{equation*}
\] where \[
\mathbf{b}^{\top} = \sum_{i=1}^N \frac{1-\pi_i^A}{\pi_i^A} \log \left(1 - \pi_i^A\right) \left(y_i - \mu_y + \Delta\right) \bx_i^{\top}\left[\operatorname{E}\left(\operatorname{H} (\btheta)\right)\right]^{-1}.
\] Decomposition for this model has following form

\[
\begin{equation*}
    \operatorname{\bA_1} = \frac{1}{N} \sum_{i=1}^N
    \left(\begin{array}{c}
        \frac{R_i\left(y_i - \mu\right)}{\pi_i^A} + \Delta\frac{R_i - \pi_i^A}{\pi_i^A}  \\
       \pi_i^A \frac{\log\left(1 - \pi_i^A\right)}{\pi_i^A} \bx_i -  R_i \frac{\log\left(1 - \pi_i^A\right)}{\pi_i^A} \bx_i
    \end{array}\right)
\end{equation*}
\] and \[
\begin{equation*}
    \operatorname{\bA_2} = \frac{1}{N} 
    \left(\begin{array}{c}
        0  \\
        \sum_{i \in \mathcal{S}_B} d_i^B \log \left(1 - \pi_i^A\right) \bx_i - \sum_{i=1}^N \pi_i^A  \frac{\log \left(1 - \pi_i^A\right)}{\pi_i^A} \bx_i
    \end{array}\right)
\end{equation*}
\] and respectively \[
\begin{equation*}
\operatorname{Var}\left(\bA_1\right) = \frac{1}{N^2} \sum_{i=1}^N
\left(\begin{array}{cc}
    \left(y_i - \mu + \Delta\right)^2 \left((\pi_i^A)^{-1} - 1\right)  & - \frac{\left(1 - \pi_i^A\right)}{\pi_i^A} \log\left(1 - \pi_i^A\right)\left(y_i - \mu + \Delta\right)\bx_i^{\top}  \\
    - \frac{\left(1 - \pi_i^A\right)}{\pi_i^A} \log\left(1 - \pi_i^A\right)\left(y_i - \mu + \Delta\right)\bx_i
    & \frac{\left(1 - \pi_i^A\right) \log^2\left(1 - \pi_i^A\right)}{\pi_i^A} \bx_i \bx_i^{\top}
\end{array}\right)
\end{equation*}
\] \[
\begin{equation*}
\operatorname{Var}\left(\bA_2\right) = 
\left(\begin{array}{cc}
     0 & \boldsymbol{0}^{\top} \\
     \boldsymbol{0} & \bD 
\end{array}\right),
\end{equation*}
\] where \[
\begin{align}
    \bD = \frac{1}{N^2}V_p\left(\sum_{i \in \mathcal{S}_B} d_i^B \log \left(1 - \pi_i^A\right) \bx_i\right) 
\end{align}
\] is variance-covariance matrix for sampling design for \(S_B\) and can
be estimated as \ldots{}

\hypertarget{mle-probit-model}{%
\subsection{MLE (probit model)}\label{mle-probit-model}}

From the model for probit regression we have following system to solve
\[
\begin{equation} 
\boldsymbol{\Phi}_n(\boldsymbol{\eta})=\left(\begin{array}{c}
\frac{1}{N} \sum_{i=1}^N\left[\frac{R_i\left(y_i-\mu\right)}{\pi_i^A}+\Delta \frac{R_i-\pi_i^A}{\pi_i^A}\right] \\
\frac{1}{N} \sum_{i=1}^N R_i \frac{\dot{\pi}_i^A}{\pi_i^A\left(1 - \pi_{i}^A\right)} \boldsymbol{x}_i-\frac{1}{N} \sum_{i \in \mathcal{S}_B} d_i^B \frac{\dot{\pi}_i^A}{1 - \pi_i^A} \boldsymbol{x}_i
\end{array}\right)=\mathbf{0},
\end{equation}
\] AS before we have to derive
\(\boldsymbol{\phi}_n(\boldsymbol{\eta})=\partial \boldsymbol{\Phi}_n(\boldsymbol{\eta}) / \partial \boldsymbol{\eta}\)
and find decomposition of \(\boldsymbol{\phi}_n(\boldsymbol{\eta})\) and
its variance-covariance matrices. Thus, we have \[
\begin{equation*}
\phi_n(\boldsymbol{\eta})=\frac{1}{N}\left(\begin{array}{cc}
-\sum_{i=1}^N\left\{\left(1-\frac{\Delta}{\mu}\right) \frac{R_i}{\pi_i^A}+\frac{\Delta}{\mu}\right\} & -\sum_{i=1}^N \frac{R_i\left(y_i-\mu + \frac{\Delta}{\mu}\right)\dot{\pi}_i^A}{\left(\pi_i^A\right)^2} \boldsymbol{x}_i^{\top} \\
\mathbf{0} & \operatorname{H}(\btheta)
\end{array}\right).
\end{equation*}
\] where \(\operatorname{H}\) is hessian corresponding to set of
parameters \(\btheta\). We can show that \[
\begin{equation*}
\left[E(\phi_n(\boldsymbol{\eta}))\right]^{-1}=\left(\begin{array}{cc}
-1 & \mathbf{b}^T\\
\mathbf{0} & \left[\frac{1}{N}\operatorname{E}\left(\operatorname{H}(\btheta)\right)\right]^{-1}
\end{array}\right),
\end{equation*}
\] where \[
\mathbf{b}^T = - \sum_{i=1}^N \frac{\left(y_i-\mu + \frac{\Delta}{\mu}\right)\dot{\pi}_i^A}{\pi_i^A}\boldsymbol{x}_i^{\top}\left[\operatorname{E}\left(\operatorname{H}(\btheta)\right)\right]^{-1}
\] and by the decomposition of \(\Phi_n(\boldsymbol{\eta})\) we get \[
\begin{equation*}
    \operatorname{\bA_1} = \frac{1}{N} \sum_{i=1}^N
    \left(\begin{array}{c}
        \frac{R_i\left(y_i - \mu\right)}{\pi_i^A} + \Delta\frac{R_i - \pi_i^A}{\pi_i^A}  \\
         R_i\frac{\dot{\pi}_i^A}{\pi_i^A\left(1 - \pi_i^A\right)} \bx_i - \pi_i^A \frac{\dot{\pi_i^A}}{\pi_i^A \left(1 - \pi_i^A\right)} \bx_i
    \end{array}\right)
\end{equation*}
\] and \[
\begin{equation*}
    \operatorname{\bA_2} = \frac{1}{N} 
    \left(\begin{array}{c}
        0  \\
        \sum_{i=1}^N \pi_i^A \frac{\dot{\pi_i^A}}{\pi_i^A\left(1 - \pi_i^A\right)} - \sum_{i \in S_B} d_i^B\frac{\dot{\pi}_i^A}{1 - \pi_i^A} \bx_i
    \end{array}\right)
\end{equation*}
\] with following variance-covariance matrices \[
\begin{equation*}
\operatorname{Var}\left(\bA_1\right) = \frac{1}{N^2} \sum_{i=1}^N
\left(\begin{array}{cc}
     \left(y_i - \mu + \Delta\right)^2 \left((\pi_i^A)^{-1} - 1\right) &     \frac{\dot{\pi}_i^A\left(y_i - \mu + \Delta\right)}{\pi_i^A}\bx_{i}^{\top}  \\
    \frac{\dot{\pi}_i^A\left(y_i - \mu + \Delta\right)}{\pi_i^A}\bx_{i} & \frac{\dot{\pi}_i^A}{\pi_i^A\left(1 - \pi_i^A\right)}\bx_i\bx_i^{\top} 
\end{array}\right)
\end{equation*}
\] and \[
\begin{equation*}
\operatorname{Var}\left(\bA_2\right) = 
\left(\begin{array}{cc}
     0 & \boldsymbol{0}^{\top} \\
     \boldsymbol{0} & \bD 
\end{array}\right),
\end{equation*}
\] where \[
\begin{align} 
    \bD = \frac{1}{N^2}V_p\left(\sum_{i \in S_B} d_i^B \frac{\dot{\pi}_i^A}{1 - \pi_i^A}\bx_i\right)
\end{align}
\] is variance covariance matrix corresponding to probability sample.

\hypertarget{gee}{%
\subsection{GEE}\label{gee}}

For generalized estimating equations we have following system of
equations to consider \[
\begin{equation}
\boldsymbol{\Phi}_n(\boldsymbol{\eta})=\left(\begin{array}{c}
\frac{1}{N} \sum_{i=1}^N\left[\frac{R_i\left(y_i-\mu\right)}{\pi_i^A}+\Delta \frac{R_i-\pi_i^A}{\pi_i^A}\right] \\
\frac{1}{N} \sum_{i=1}^N R_i \operatorname{h}\left(\bx_i\right)-\frac{1}{N} \sum_{i \in \mathcal{S}_B} d_i^B \pi_i^A \operatorname{h}\left(\bx_i\right)
\end{array}\right)=\mathbf{0}
\end{equation}
\] We have following decomposition of
\(\boldsymbol{\Phi}_n(\boldsymbol{\eta})\) for this approach \[
\begin{equation}
\mathbf{A}_1=\frac{1}{N} \sum_{i=1}^N\left(\begin{array}{c}
\frac{R_i\left(y_i-\mu\right)}{\pi_i^A}+\Delta \frac{R_i-\pi_i^A}{\pi_i^A} \\
R_i \operatorname{h}\left(\bx_i\right)
\end{array}\right), \quad \mathbf{A}_2=\frac{1}{N}\left(\begin{array}{c}
0 \\
- \sum_{i \in S_B} d_i^B \pi_i^A \operatorname{h}\left(\bx_i\right)
\end{array}\right)
\end{equation}
\] It can be shown that\\
1. for \(\operatorname{h}\left(\bx_i\right) = \bx_i\)

\[
\mathbf{V}_1=\frac{1}{N^2} \sum_{i=1}^N\left(\begin{array}{cc}
\left\{\left(1-\pi_i^A\right) / \pi_i^A\right\}\left(y_i-\mu+\Delta\right)^2 & \left(1-\pi_i^A\right)\left(y_i-\mu+\Delta\right) \boldsymbol{x}_i^{\top} \\
\left(1-\pi_i^A\right)\left(y_i-\mu+\Delta\right) \boldsymbol{x}_i & \pi_i^A\left(1-\pi_i^A\right) \boldsymbol{x}_i \boldsymbol{x}_i^{\top}
\end{array}\right)
\] and \[
\mathbf{V}_2=\left(\begin{array}{cc}
0 & \mathbf{0}^{\top} \\
\mathbf{0} & \mathbf{D}
\end{array}\right)
\] where
\[\mathbf{D}=N^{-2} V_p\left(\sum_{i \in \mathcal{S}_B} d_i^B \pi_i^A \boldsymbol{x}_i\right)\]\\
2. for
\(\operatorname{h}\left(\bx_i\right) = \bx_i \pi_i^A(\bx_i, \btheta) ^{-1}\)
\[
\mathbf{V}_1=\frac{1}{N^2} \sum_{i=1}^N\left(\begin{array}{cc}
\left\{\left(1-\pi_i^A\right) / \pi_i^A\right\}\left(y_i-\mu+\Delta\right)^2 & \frac{\left(1-\pi_i^A\right)}{\pi_i^A}\left(y_i-\mu+\Delta\right) \boldsymbol{x}_i^{\top} \\
\frac{\left(1-\pi_i^A\right)}{\pi_i^A}\left(y_i-\mu+\Delta\right) \boldsymbol{x}_i & \left(1-\pi_i^A\right) \boldsymbol{x}_i \boldsymbol{x}_i^{\top}
\end{array}\right)
\] and \[
\mathbf{V}_2=\left(\begin{array}{cc}
0 & \mathbf{0}^{\top} \\
\mathbf{0} & \mathbf{D}
\end{array}\right)
\] where
\[\mathbf{D}=N^{-2} V_p\left(\sum_{i \in \mathcal{S}_B} d_i^B \boldsymbol{x}_i\right).\]
To determine the variance-covariance matrix for
\(\hat{\boldsymbol{\eta}}\) we need to derive
\(\boldsymbol{\phi}_n(\boldsymbol{\eta})=\partial \boldsymbol{\Phi}_n(\boldsymbol{\eta}) / \partial \boldsymbol{\eta}\)
which depends on the choosen \(h\)-function and model for propensity
score.

\hypertarget{dr-estimator-variance}{%
\section{DR estimator variance}\label{dr-estimator-variance}}

In this subsection we show how to derive variance for doubly robust
estimator under probit and cloglog models for propensity score and
General estimating equations as well. The derivation in analogous to
that of the section on doubly robust methods for non-probability sample.
The technical difference lies in the imputation of another modeling
function for propensity score.

\hypertarget{mle-with-cloglog-and-probit-models}{%
\subsection{MLE with cloglog and probit
models}\label{mle-with-cloglog-and-probit-models}}

Using the equation (2.1) on \(\mu_{IPW2}\) for the probit model we have
\[
\begin{equation}
\begin{aligned}
\frac{1}{\hat{N}^A} \sum_{i=1}^N \frac{R_i\left\{y_i-m\left(\boldsymbol{x}_i, \boldsymbol{\beta}^*\right)\right\}}{\hat{\pi}_i^A} = & h_N+\frac{1}{N} \sum_{i=1}^N R_i\left\{\frac{y_i-m\left(\boldsymbol{x}_i, \boldsymbol{\beta}^*\right)-h_N}{\pi_i^A}-\mathbf{b}^{\top} \frac{\dot{\pi}_i^A}{\pi_i^A \left(1 - \pi_i^A\right)} \boldsymbol{x}_i\right\} \\
& +\mathbf{b}_3^{\top} \frac{1}{N} \sum_{i \in \mathcal{S}_B} d_i^B \frac{\dot{\pi}_i^A}{\left(1 - \pi_i^A\right)} \boldsymbol{x}_i+o_p\left(n_A^{-1 / 2}\right),
\end{aligned}
\end{equation}
\] where
\(h_n = \frac{1}{N} \sum_{i=1}^N \left(y_i - \operatorname{m}\left(\bx_i,\bbeta^*\right)\right)\)
and \[
\mathbf{b}^{\top} = -\sum_{i=1}^N \frac{\left(y_i - \operatorname{m}\left(\bx_i,\bbeta^*\right) - h_n\right)\dot{\pi}_i^A}{\pi_i^A}\boldsymbol{x}_i^{\top}\left[\operatorname{E}\left(\operatorname{H}(\btheta)\right)\right]^{-1}.
\] The second part of the estimator has the following expansion \[
\begin{equation} \label{eq2}
\frac{1}{\hat{N}^B} \sum_{i \in \mathcal{S}_B} d_i^B m\left(\bx_i,\bbeta^*\right)=\frac{1}{N} \sum_{i=1}^N m\left(\bx_i,\bbeta^*\right)+\frac{1}{N} \sum_{i \in \mathcal{S}_B} d_i^B\left\{m\left(\bx_i,\bbeta^*\right)-\frac{1}{N} \sum_{i=1}^N m\left(\bx_i,\bbeta^*\right)\right\}+O_p\left(n_B^{-1}\right)
\end{equation}
\] Putting these two parts together implies: \[
\begin{equation}
\begin{aligned}
\hat{\mu}_{D R 2}-\mu_y = & \frac{1}{N} \sum_{i=1}^N R_i\left\{\frac{y_i-m\left(\boldsymbol{x}_i, \boldsymbol{\beta}^*\right)-h_N}{\pi_i^A}-\mathbf{b}_3^{\top} \frac{\dot{\pi}_i^A}{\pi_i^A \left(1 - \pi_i^A \right)} \boldsymbol{x}_i\right\} \\
& +\frac{1}{N} \sum_{i \in \mathcal{S}_B} d_i^B t_i +o_p\left(n_A^{-1 / 2}\right)
\end{aligned}
\end{equation}
\] It follows that \[
\begin{equation*}
    \begin{aligned}
        \text{var}\left(\hat{\mu}_{DR}\right) = & \frac{1}{N^2} \sum_{i=1}^N \left(1 - \pi_i^A\right)\pi_i^A \left[\frac{y_i - \operatorname{m}\left(\bx_i,\bbeta^*\right) - h_n}{\pi_i^A} - \mathbf{b}_3^T\frac{\dot{\pi}_i^A}{\pi_i^A \left(1 - \pi_i^A \right)}\bx_i\right]^2 \\ & + \operatorname{W} + o(n_A^{-1}),
    \end{aligned}
\end{equation*}
\] where \[
\begin{align}
    \operatorname{W} = \frac{1}{N^2}\text{var}_p\left(\sum_{i \in S_B} d_i^B t_i\right)
\end{align}
\] and
\(t_i = \frac{\dot{\pi}_i^A}{1 - \pi_i^A} \bx_i^{\top} \mathbf{b}_3 + \operatorname{m}\left(\bx_i,\bbeta^*\right)- \frac{1}{N} \sum_{i=1}^N \operatorname{m}\left(\bx_i,\bbeta^*\right)\).

When propensity scores are estimating by complementary log-log model,
the results are as follows \[
\begin{equation}
\begin{aligned}
\frac{1}{\hat{N}^A} \sum_{i=1}^N \frac{R_i\left\{y_i-m\left(\boldsymbol{x}_i, \boldsymbol{\beta}^*\right)\right\}}{\hat{\pi}_i^A} = & h_N+\frac{1}{N} \sum_{i=1}^N R_i\left\{\frac{y_i-m\left(\boldsymbol{x}_i, \boldsymbol{\beta}^*\right)-h_N}{\pi_i^A}-\mathbf{b}^{\top} \frac{\log \left(1 - \pi_i^A\right)}{\pi_i^A} \boldsymbol{x}_i\right\} \\
& +\mathbf{b}_3^{\top} \frac{1}{N} \sum_{i \in \mathcal{S}_B} d_i^B \log \left(1 - \pi_i^A\right) \boldsymbol{x}_i+o_p\left(n_A^{-1 / 2}\right),
\end{aligned}
\end{equation}
\] where
\[\mathbf{b}^{\top} = -\sum_{i=1}^N \frac{\left(y_i - \operatorname{m}\left(\bx_i,\bbeta^*\right) - h_n\right)\left(1 - \pi_i^A\right) \log \left(1 - \pi_i^A\right)}{\pi_i^A}\boldsymbol{x}_i^{\top}\left[\operatorname{E}\left(\operatorname{H}_{l}(\btheta)\right)\right]^{-1}.\]
The second part of the \(\hat{\mu}_{DR}\) is expanded as in (2.6) \[
\begin{equation}
\begin{aligned}
\hat{\mu}_{D R 2}-\mu_y = & \frac{1}{N} \sum_{i=1}^N R_i\left\{\frac{y_i-m\left(\boldsymbol{x}_i, \boldsymbol{\beta}^*\right)-h_N}{\pi_i^A}-\mathbf{b}_3^{\top} \frac{\log \left(1 - \pi_i^A\right)}{\pi_i^A} \boldsymbol{x}_i\right\} \\
& +\frac{1}{N} \sum_{i \in \mathcal{S}_B} d_i^B t_i +o_p\left(n_A^{-1 / 2}\right),
\end{aligned}
\end{equation}
\] what implies that \[
\begin{equation*}
    \begin{aligned}
        \operatorname{Var}\left(\hat{\mu}_{DR}\right) = & \frac{1}{N^2} \sum_{i=1}^N \left(1 - \pi_i^A\right)\pi_i^A \left[\frac{y_i - \operatorname{m}\left(\bx_i,\bbeta^*\right) - h_n}{\pi_i^A} - \mathbf{b}_3^T \frac{\log\left(1 - \pi_i^A\right)}{\pi_i^A}\bx_i\right]^2 \\ & + 
        \operatorname{W} + o(n_A^{-1}),
    \end{aligned}
\end{equation*}
\] where \[
\begin{align} 
    \operatorname{W} = \frac{1}{N^2}\text{var}_p\left(\sum_{i \in S_B} d_i^B t_i\right)
\end{align}
\] and
\(t_i = \log \left(1 - \pi_i^A \right) \bx_i^{\top} \mathbf{b}_3 + \operatorname{m}\left(\bx_i,\bbeta^*\right)- \frac{1}{N} \sum_{i=1}^N \operatorname{m}\left(\bx_i,\bbeta^*\right)\).

\hypertarget{gee-1}{%
\subsection{GEE}\label{gee-1}}

As with MLE, we will use Taylor expansion to derive the variance for the
DR of the estimator. For \(\operatorname{h}\left(\bx_i\right) = \bx_i\)
we have \[
\begin{aligned}
\frac{1}{\hat{N}^A} \sum_{i=1}^N \frac{R_i\left\{y_i-m\left(\boldsymbol{x}_i, \boldsymbol{\beta}^*\right)\right\}}{\hat{\pi}_i^A}= & h_N+\frac{1}{N} \sum_{i=1}^N R_i\left\{\frac{y_i-m\left(\boldsymbol{x}_i, \boldsymbol{\beta}^*\right)-h_N}{\pi_i^A}-\mathbf{b}^{\top} \boldsymbol{x}_i\right\} \\
& +\mathbf{b}^{\top} \frac{1}{N} \sum_{i \in \mathcal{S}_B} d_i^B \pi_i^A \boldsymbol{x}_i+o_p\left(n_A^{-1 / 2}\right)
\end{aligned}
\] where
\(h_N=N^{-1} \sum_{i=1}^N\left\{y_i-m\left(\boldsymbol{x}_i, \boldsymbol{\beta}^*\right)\right\}\).

The second part of \(\hat{\mu}_{D R 2}\) given in is the Hájek estimator
under the probability sampling design for \(\mathcal{S}_B\), which has
the following expansion \[
\frac{1}{\hat{N}^B} \sum_{i \in \mathcal{S}_B} d_i^B m_i=\frac{1}{N} \sum_{i=1}^N m_i+\frac{1}{N} \sum_{i \in \mathcal{S}_B} d_i^B\left\{m_i-\frac{1}{N} \sum_{i=1}^N m_i\right\}+O_p\left(n_B^{-1}\right)
\], where \(m_i=m\left(\boldsymbol{x}_i, \boldsymbol{\beta}^*\right)\).
Putting these two parts together we have \[
\hat{\mu}_{D R 2}-\mu_y=\frac{1}{N} \sum_{i=1}^N R_i\left\{\frac{y_i-m\left(\boldsymbol{x}_i, \boldsymbol{\beta}^*\right)-h_N}{\pi_i^A}-\mathbf{b}^{\top} \boldsymbol{x}_i\right\}+\frac{1}{N} \sum_{i \in \mathcal{S}_B} d_i^B t_i+o_p\left(n_A^{-1 / 2}\right)
\] where
\(t_i=\pi_i^A \boldsymbol{x}_i^{\top} \mathbf{b}_3+m\left(\boldsymbol{x}_i, \boldsymbol{\beta}^*\right)-N^{-1} \sum_{i=1}^N m\left(\boldsymbol{x}_i, \boldsymbol{\beta}^*\right)\)
and it follows that \[
\begin{gathered}
\text{var}_{\mathrm{DR} 2}=\frac{1}{N^2} \sum_{i=1}^N\left(1-\pi_i^{\mathrm{A}}\right) \pi_i^{\mathrm{A}}\left[\left\{\frac{y_i-m\left(\boldsymbol{x}_i, \boldsymbol{\beta}^*\right)-h_{\mathrm{N}}}{\pi_i^A}\right\} -
\mathbf{b}^{\top} \boldsymbol{x}_i\right]^2 + N^{-2} V_p\left(\sum_{i \in \mathcal{S}_{\mathrm{B}}} d_i^{\mathrm{B}} t_i\right)
\end{gathered}
\]

By analogy, it can be shown that for
\(\operatorname{h}\left(\bx_i\right) = \bx_i \pi_i^A\left(\bx_i^{\mathrm{T}} \btheta \right)^{-1}\)
we have \[
\begin{gathered}
\text{var}_{\mathrm{DR} 2}=\frac{1}{N^2} \sum_{i=1}^N\left(1-\pi_i^{\mathrm{A}}\right) \pi_i^{\mathrm{A}}\left[\left\{\frac{y_i-m\left(\boldsymbol{x}_i, \boldsymbol{\beta}^*\right)-h_{\mathrm{N}}}{\pi_i^A}\right\} -
\frac{\mathbf{b}_3^{\top} \boldsymbol{x}_i}{\pi_i^A}\right]^2 + N^{-2} V_p\left(\sum_{i \in \mathcal{S}_{\mathrm{B}}} d_i^{\mathrm{B}} t_i\right)
\end{gathered}
\] where
\(t_i= \boldsymbol{x}_i^{\top} \mathbf{b}_3+m\left(\boldsymbol{x}_i, \boldsymbol{\beta}^*\right)-N^{-1} \sum_{i=1}^N m\left(\boldsymbol{x}_i, \boldsymbol{\beta}^*\right)\)
and independently of the function of \(\operatorname{h}\)

\begin{verbatim}
1. for logit model
\end{verbatim}

\[
\mathbf{b}^{\top}=\left[\frac{1}{N} \sum_{i=1}^N\left(1-\pi_i^A\right)\left\{y_i-m\left(\boldsymbol{x}_i, \boldsymbol{\beta}^*\right)-h_N\right\} \boldsymbol{x}_i^{\top}\right]\left\{\frac{1}{N} \sum_{i=1}^N \pi_i^A\left(1-\pi_i^A\right) \boldsymbol{x}_i \boldsymbol{x}_i^{\top}\right\}^{-1}
\]

\begin{verbatim}
2. for complementary loglog model
\end{verbatim}

\[\mathbf{b}^{\top} = -\sum_{i=1}^N \frac{\left(y_i - \operatorname{m}\left(\bx_i,\bbeta^*\right) - h_n\right)\left(1 - \pi_i^A\right) \log \left(1 - \pi_i^A\right)}{\pi_i^A}\boldsymbol{x}_i^{\top} \left\{\frac{1}{N}\sum_{i=1}^N \left(1 - \pi_i^A\right) \exp\left(\bx_i^{\top}\btheta\right) \bx_i \bx_i^{\top}\right\} ^ {-1}\]

\begin{verbatim}
3. for probit model
\end{verbatim}

\[\mathbf{b}^{\top} = -\sum_{i=1}^N \frac{\left(y_i - \operatorname{m}\left(\bx_i,\bbeta^*\right) - h_n\right)\dot{\pi}_i^A}{\pi_i^A}\boldsymbol{x}_i^{\top} \left\{\frac{1}{N}\sum_{i=1}^N \pi_i^A \bx_i \bx_i^{\top}\right\} ^ {-1}\]

\bookmarksetup{startatroot}

\hypertarget{references}{%
\chapter{References}\label{references}}

\hypertarget{refs}{}
\begin{CSLReferences}{1}{0}
\leavevmode\vadjust pre{\hypertarget{ref-kim20}{}}%
Jae Kwang Kim, Yilin Chen, Seho Park. 2020. {``Combining Nonprobability
and Probability Survey Samples Through Mass Imputation.''} \emph{J R
Stat Soc Series} 184: 941--63.

\leavevmode\vadjust pre{\hypertarget{ref-kim22}{}}%
Kim, Jae Kwang. 2022. {``A Gentle Introduction to Data Integration in
Survey Sampling.''} \emph{The Survey Statistician} 85: 19--29.

\leavevmode\vadjust pre{\hypertarget{ref-shu20}{}}%
Shu Yang, Rui Song, Jae Kwang Kim. 2020. {``Doubly Robust Inference When
Combining Probability and Nonprobability Samples with High Dimensional
Data.''} \emph{J. R. Statist. Soc. B}.

\leavevmode\vadjust pre{\hypertarget{ref-chen20}{}}%
Yilin Chen, Changbao Wu, Pengfei Li. 2020. {``Doubly Robust Inference
with Nonprobability Survey Samples.''} \emph{Journal of the American
Statistical Association} 115 (532): 2011--21.
\url{https://doi.org/10.1080/01621459.2019.1677241}.

\end{CSLReferences}

(Yilin Chen 2020) (Jae Kwang Kim 2020) (Kim 2022) (Shu Yang 2020)



\end{document}
