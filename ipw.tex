% Options for packages loaded elsewhere
\PassOptionsToPackage{unicode}{hyperref}
\PassOptionsToPackage{hyphens}{url}
\PassOptionsToPackage{dvipsnames,svgnames,x11names}{xcolor}
%
\documentclass[
  letterpaper,
  DIV=11,
  numbers=noendperiod]{scrreprt}

\usepackage{amsmath,amssymb}
\usepackage{lmodern}
\usepackage{iftex}
\ifPDFTeX
  \usepackage[T1]{fontenc}
  \usepackage[utf8]{inputenc}
  \usepackage{textcomp} % provide euro and other symbols
\else % if luatex or xetex
  \usepackage{unicode-math}
  \defaultfontfeatures{Scale=MatchLowercase}
  \defaultfontfeatures[\rmfamily]{Ligatures=TeX,Scale=1}
\fi
% Use upquote if available, for straight quotes in verbatim environments
\IfFileExists{upquote.sty}{\usepackage{upquote}}{}
\IfFileExists{microtype.sty}{% use microtype if available
  \usepackage[]{microtype}
  \UseMicrotypeSet[protrusion]{basicmath} % disable protrusion for tt fonts
}{}
\makeatletter
\@ifundefined{KOMAClassName}{% if non-KOMA class
  \IfFileExists{parskip.sty}{%
    \usepackage{parskip}
  }{% else
    \setlength{\parindent}{0pt}
    \setlength{\parskip}{6pt plus 2pt minus 1pt}}
}{% if KOMA class
  \KOMAoptions{parskip=half}}
\makeatother
\usepackage{xcolor}
\setlength{\emergencystretch}{3em} % prevent overfull lines
\setcounter{secnumdepth}{-\maxdimen} % remove section numbering
% Make \paragraph and \subparagraph free-standing
\ifx\paragraph\undefined\else
  \let\oldparagraph\paragraph
  \renewcommand{\paragraph}[1]{\oldparagraph{#1}\mbox{}}
\fi
\ifx\subparagraph\undefined\else
  \let\oldsubparagraph\subparagraph
  \renewcommand{\subparagraph}[1]{\oldsubparagraph{#1}\mbox{}}
\fi


\providecommand{\tightlist}{%
  \setlength{\itemsep}{0pt}\setlength{\parskip}{0pt}}\usepackage{longtable,booktabs,array}
\usepackage{calc} % for calculating minipage widths
% Correct order of tables after \paragraph or \subparagraph
\usepackage{etoolbox}
\makeatletter
\patchcmd\longtable{\par}{\if@noskipsec\mbox{}\fi\par}{}{}
\makeatother
% Allow footnotes in longtable head/foot
\IfFileExists{footnotehyper.sty}{\usepackage{footnotehyper}}{\usepackage{footnote}}
\makesavenoteenv{longtable}
\usepackage{graphicx}
\makeatletter
\def\maxwidth{\ifdim\Gin@nat@width>\linewidth\linewidth\else\Gin@nat@width\fi}
\def\maxheight{\ifdim\Gin@nat@height>\textheight\textheight\else\Gin@nat@height\fi}
\makeatother
% Scale images if necessary, so that they will not overflow the page
% margins by default, and it is still possible to overwrite the defaults
% using explicit options in \includegraphics[width, height, ...]{}
\setkeys{Gin}{width=\maxwidth,height=\maxheight,keepaspectratio}
% Set default figure placement to htbp
\makeatletter
\def\fps@figure{htbp}
\makeatother

\KOMAoption{captions}{tableheading}
\makeatletter
\makeatother
\makeatletter
\makeatother
\makeatletter
\@ifpackageloaded{caption}{}{\usepackage{caption}}
\AtBeginDocument{%
\ifdefined\contentsname
  \renewcommand*\contentsname{Table of contents}
\else
  \newcommand\contentsname{Table of contents}
\fi
\ifdefined\listfigurename
  \renewcommand*\listfigurename{List of Figures}
\else
  \newcommand\listfigurename{List of Figures}
\fi
\ifdefined\listtablename
  \renewcommand*\listtablename{List of Tables}
\else
  \newcommand\listtablename{List of Tables}
\fi
\ifdefined\figurename
  \renewcommand*\figurename{Figure}
\else
  \newcommand\figurename{Figure}
\fi
\ifdefined\tablename
  \renewcommand*\tablename{Table}
\else
  \newcommand\tablename{Table}
\fi
}
\@ifpackageloaded{float}{}{\usepackage{float}}
\floatstyle{ruled}
\@ifundefined{c@chapter}{\newfloat{codelisting}{h}{lop}}{\newfloat{codelisting}{h}{lop}[chapter]}
\floatname{codelisting}{Listing}
\newcommand*\listoflistings{\listof{codelisting}{List of Listings}}
\makeatother
\makeatletter
\@ifpackageloaded{caption}{}{\usepackage{caption}}
\@ifpackageloaded{subcaption}{}{\usepackage{subcaption}}
\makeatother
\makeatletter
\@ifpackageloaded{tcolorbox}{}{\usepackage[many]{tcolorbox}}
\makeatother
\makeatletter
\@ifundefined{shadecolor}{\definecolor{shadecolor}{rgb}{.97, .97, .97}}
\makeatother
\makeatletter
\makeatother
\ifLuaTeX
  \usepackage{selnolig}  % disable illegal ligatures
\fi
\IfFileExists{bookmark.sty}{\usepackage{bookmark}}{\usepackage{hyperref}}
\IfFileExists{xurl.sty}{\usepackage{xurl}}{} % add URL line breaks if available
\urlstyle{same} % disable monospaced font for URLs
\hypersetup{
  colorlinks=true,
  linkcolor={blue},
  filecolor={Maroon},
  citecolor={Blue},
  urlcolor={Blue},
  pdfcreator={LaTeX via pandoc}}

\author{}
\date{}

\begin{document}
\ifdefined\Shaded\renewenvironment{Shaded}{\begin{tcolorbox}[borderline west={3pt}{0pt}{shadecolor}, enhanced, interior hidden, breakable, sharp corners, frame hidden, boxrule=0pt]}{\end{tcolorbox}}\fi

\hypertarget{inverse-probability-weighting}{%
\chapter{Inverse probability
weighting}\label{inverse-probability-weighting}}

\newcommand{\bSigma}{\boldsymbol{\Sigma}}
\newcommand{\bOmega}{\boldsymbol{\Omega}}
\newcommand{\bTheta}{\boldsymbol{\Theta}}
\newcommand{\bPi}{\boldsymbol{\Pi}}
\newcommand{\bbeta}{\boldsymbol{\beta}}
\newcommand{\balpha}{\boldsymbol{\alpha}}
\newcommand{\brho}{\boldsymbol{\rho}}
\newcommand{\beps}{\boldsymbol{\epsilon}}
\newcommand{\blambda}{\boldsymbol{\lambda}}
\newcommand{\bgamma}{\boldsymbol{\gamma}}
\newcommand{\btheta}{\boldsymbol{\theta}}
\newcommand{\bmu}{\boldsymbol{\mu}}
\newcommand{\bpi}{\boldsymbol{\pi}}
\newcommand{\bphi}{\boldsymbol{\phi}}
\newcommand{\bPhi}{\boldsymbol{\Phi}}
\newcommand{\boldeta}{\boldsymbol{\eta}}
\newcommand{\bx}{\boldsymbol{x}}
\newcommand{\bD}{\boldsymbol{D}}
\newcommand{\bV}{\boldsymbol{V}}
\newcommand{\bv}{\boldsymbol{v}}
\newcommand{\bY}{\boldsymbol{Y}}
\newcommand{\bA}{\boldsymbol{A}}
\newcommand{\bB}{\boldsymbol{B}}
\newcommand{\bR}{\boldsymbol{R}}
\newcommand{\bM}{\boldsymbol{M}}
\newcommand{\bI}{\boldsymbol{I}}
\newcommand{\bC}{\boldsymbol{C}}
\newcommand{\bW}{\boldsymbol{W}}
\newcommand{\bw}{\boldsymbol{w}}
\newcommand{\bd}{\boldsymbol{d}}
\newcommand{\bT}{\boldsymbol{T}}
\newcommand{\bt}{\boldsymbol{t}}
\newcommand{\bZ}{\boldsymbol{Z}}
\newcommand{\bX}{\boldsymbol{X}}
\newcommand{\bz}{\boldsymbol{z}}
\newcommand{\by}{\boldsymbol{y}}
\newcommand{\br}{\boldsymbol{r}}
\newcommand{\bp}{\boldsymbol{p}}
\newcommand{\bb}{\boldsymbol{b}}
\newcommand{\bZero}{\boldsymbol{0}}
\newcommand{\bOne}{\boldsymbol{1}}

\hypertarget{motivation-and-assumptions}{%
\section{Motivation and assumptions}\label{motivation-and-assumptions}}

Let \(\mathcal{U}=\{1,2, \ldots, N\}\) represent the finite population
with N units and
\(\left\{\left(\bx_i, y_i\right), i \in \mathcal{S}_{\mathrm{A}}\right\}\)
and \(\{\left(\bx_i, d_i^B), i \in \mathcal{S}_{\mathrm{B}}\right\}\) be
the datasets from non-probability and probability samples respectively.
Following assumptions are required for this model:

\begin{enumerate}
\def\labelenumi{\arabic{enumi}.}
\item
  The selection indicator \(R_i\) and the response variable \(y_i\) are
  independent given the set of covariates \(x_i\).
\item
  All units have a nonzero propensity score, that is, \(\pi_i^A > 0\)
  for all \(i\).
\item
  The indicator variables \(R_i^A\) and \(R_j^A\) are independent for
  given \(x_i\) and \(x_j\) for \(i \neq j\).
\end{enumerate}

\hypertarget{maximum-likelihood-estimation}{%
\section{Maximum likelihood
estimation}\label{maximum-likelihood-estimation}}

Suppose that propensity score can be modelled parametrically as
\(\mathbb{P}\left(R_i=1 \mid \bx_i\right) = \pi(\bx_{i}, \btheta_{0})\).
The maximum likelihood estimator is computed as
\(\hat{\pi}_{i}^{A} = \pi(\bx_{i}, \hat{\btheta}_{0})\), where
\(\hat{\btheta}_{0}\) is the maximizer of the following log-likelihood
function:

\[
\begin{align}
    \begin{split}
 \ell(\boldsymbol{\theta}) & =\sum_{i=1}^N\left\{R_i \log \pi_i^{\mathrm{A}}+\left(1-R_i\right) \log \left(1-\pi_i^{\mathrm{A}}\right)\right\} \\ & =\sum_{i \in \mathcal{S}_{\mathrm{A}}} \log \left\{\frac{\pi\left(\boldsymbol{x}_i, \boldsymbol{\theta}\right)}{1-\pi\left(\boldsymbol{x}_i, \boldsymbol{\theta}\right)}\right\}+\sum_{i=1}^N \log \left\{1-\pi\left(\boldsymbol{x}_i, \boldsymbol{\theta}\right)\right\}
    \end{split}
\end{align}
\]

Since we do not observe \(\bx_i\) for all units, Yilin Chen, Pengfei Li
\& Changbao Wu presented following log-likelihood function is subject to
data integration basing on samples \(S_A\) and \(S_B\). They proposed
logistic regression model with
\(\pi(\bx_{i}, \btheta) = \frac{\exp(\bx_{i}^{\top}\btheta)}{\exp(\bx_{i}^{\top}\btheta) + 1}\)
in order to estimate \(\btheta\). We expanded this approach on probit
regression and complementary log-log model. For the sake of accuracy,
let us recall that the probit and cloglog models are based on the
assumption that model takes the form
\(\pi(\bx_{i},\btheta) = \Phi(\bx_{i}^{\top}\btheta)\) and
\(\pi(\bx_{i}, \btheta) = 1 - \exp(-\exp(\bx_{i}^{\top}\btheta))\)
respectively.

\[
\begin{align}
    \ell^{*}(\btheta) = \sum_{i \in S_{A}}\log \left\{\frac{\pi(\bx_{i}, \btheta)}{1 - \pi(\bx_{i},\btheta)}\right\} + \sum_{i \in S_{B}}d_{i}^{B}\log\{1 - \pi({\bx_{i},\btheta})\}
\end{align}
\] In the following subsections we present the full derivation of the
MLE, depending on the assumed model for the propensity score.

\hypertarget{logistic-regression}{%
\subsection{Logistic regression}\label{logistic-regression}}

Log-likelihood function with logistic regression is given by \[
\ell^{*}(\btheta) = \sum_{i \in S_A}\bx_{i}^{\top}\btheta - \sum_{i \in S_B}d_{i}^{B}\log\{1 + \exp(\bx_{i}^{\top}\btheta)\}
\] with analytical gradient and hessian given by \[
\frac{\partial \ell^*}{\partial\btheta} = \sum_{i \in S_{A}}\bx_{i} - \sum_{i \in S_{B}}d_{i}^{B}\pi(\bx_{i}, \btheta)\bx_{i}
\] and \[
    \frac{\partial^{2} \ell^{*}}{\partial\btheta^{T} \partial\btheta} =- \sum_{i \in S_B}d_i^B\pi(\bx_i,\btheta)(1 - \pi(\bx_i,\btheta))\bx_i\bx_i^{\top} = \bX_B^{\top}\operatorname{\bW}_{B}\bX_B,
\] respectively, where \[
\begin{align*}
    \operatorname{\bW}_{B} =
    diag & \left(-d_1^B\pi(\bx_{1},\btheta)(1 - \pi(\bx_{1},\btheta)), -d_2^B\pi(\bx_{2},\btheta)(1 - \pi(\bx_{2},\btheta)), \right. \\
     & \left. \ldots, -d_{n_{B}}^{B}\pi(\bx_{n_{B}},\btheta)(1 - \pi(x_{n_{B}},\btheta))\right).
\end{align*}
\]

\hypertarget{complementary-log-log-regression}{%
\subsection{Complementary log-log
regression}\label{complementary-log-log-regression}}

Similarly, log-likelihood function has form of \[
\ell^{*}(\btheta) = \sum_{i \in S_{A}}\left\{\log\{1 - \exp(-\exp(\bx_{i}^{\top}\btheta))\} + \exp(\bx_{i}^{\top}\btheta)\right\} - \sum_{i \in S_{B}} d_{i}^{B}\exp(\bx_{i}^{\top}\btheta)
\] with analytical gradient and hessian equal to \[
    \frac{\partial \ell^*}{\partial\btheta} = \sum_{i \in S_{A}}\frac{\exp(\bx_{i}^{\top}\btheta)\bx_{i}}{\pi(\bx_{i}, \btheta)} - \sum_{i \in S_{B}}d_{i}^{B}\exp(\bx_{i}^{T}\btheta)\bx_{i}
\] and \[
\begin{align*}
    \begin{split}
    \frac{\partial^{2} \ell^{*}}{\partial\btheta^{T} \partial\btheta} & = \sum_{i \in S_A} \frac{\exp(\bx_{i}^{\top}\btheta)}{\pi(\bx_{i}, \btheta)} \left\{1 - \frac{\exp(\bx_{i}^{\top}\btheta)}{\pi(\bx_{i}, \btheta)} + \exp(\bx_{i}^{\top}\btheta)\right\}\bx_i\bx_i^{\top} - \sum_{i \in S_B}d_i^B\exp (\bx_{i}^{\top}\btheta)\bx_i\bx_i^{\top} \\ & = \bX_A^{\top}\operatorname{\bW}_{Ac}\bX_A - \bX_B^{\top}\operatorname{\bW}_{Bc}\bX_B,
    \end{split}
\end{align*}
\] respectively, where \[
\begin{align*}
    \operatorname{\bW}_{Ac} =  Diag & \left(\frac{\exp(\bx_{1}^{\top}\btheta)}{\pi(\bx_{1}, \btheta)} \left\{1 - \frac{\exp(\bx_{1}^{\top}\btheta)}{\pi(\bx_{1}, \btheta)} + \exp(\bx_{1}^{\top}\btheta)\right\}, \right.
    \\
    & \left. \frac{\exp(\bx_{2}^{\top}\btheta)}{\pi(\bx_{2}, \btheta)} \left\{1 - \frac{\exp(\bx_{2}^{\top}\btheta)}{\pi(\bx_{2}, \btheta)} + \exp(\bx_{2}^{\top}\btheta)\right\}, \right.
    \\
    & \left. \ldots, \right.
    \\ 
    & \left. \frac{\exp(\bx_{n_A}^{\top}\btheta)} {\pi(\bx_{n_A}, \btheta)} \left\{1 - \frac{\exp(\bx_{n_A}^{\top}\btheta)}{\pi(\bx_{n_A}, \btheta)} + \exp(\bx_{n_A}^{\top}\btheta)\right\} \right)
\end{align*}
\] and \[
\begin{align*}
    \operatorname{\bW}_{Bc} = Diag \left(d_1^B\exp (\bx_{1}^{\top}\btheta), d_2^B\exp (\bx_{2}^{\top}\btheta), \ldots, d_{n_B}^B\exp (\bx_{n_{B}}^{\top}\btheta)\right).
\end{align*}
\]

\hypertarget{probit-regression}{%
\subsection{Probit regression}\label{probit-regression}}

For probit model calculations are as follow \[
\begin{align*}
    \ell^{*}(\btheta) & = \sum_{i \in S_{A}}\log\left\{\frac{\Phi(\bx_{i}^{\top}\btheta)}{1 - \Phi(\bx_{i}^{\top}\btheta)}\right\} + \sum_{i \in S_{B}}d_{i}^{B}\log\{1 - \Phi(\bx_{i}^{\top}\btheta)\}
\end{align*}
\] with analytical gradient as \[
\begin{align*}
        \frac{\partial \ell^*}{\partial\btheta} = \sum_{i \in S_A}\frac{\phi(\bx_i^{\top}\btheta)}{\Phi(\bx_i^{\top}\btheta)(1 - \Phi(\bx_i^{\top}\btheta))}\bx_i - \sum_{i \in S_B}d_i^B\frac{\phi(\bx_i^{\top}\btheta)}{1 - \Phi(\bx_i^{\top}\btheta)}\bx_i.
\end{align*}
\] \(\hat{\btheta}\) can be found by using the following
Netwon-Raphson's iterative method: \[
\btheta^{(m)} = \btheta^{(m-1)} - \{H(\btheta^{(m-1)}\}^{-1}U(\btheta^{(m-1})),
\] where \(\operatorname{H}\) - hessian, \(\operatorname{U}\) -
gradient.

\hypertarget{general-estimating-equations}{%
\section{General estimating
equations}\label{general-estimating-equations}}

The pseudo score equations derived from Maximum Likelihood Estimation
methods may be replaced by a system of general estimating equations. Let
\(\operatorname{h}\left(\bx\right)\) be the smooth function and \[
\begin{equation}
\mathbf{U}(\btheta)=\sum_{i \in S_A} \mathbf{h}\left(\mathbf{x}_i, \btheta\right)-\sum_{i \in S_B} d_i^B \pi\left(\mathbf{x}_i, \btheta\right) \mathbf{h}\left(\mathbf{x}_i, \btheta\right).
\end{equation}
\] Under \(\operatorname{h}\left(\bx_i\right) = \bx_i\) and logistic
model for propensity score, Equation (2.1) looks like disorted version
of the score equation from MLE method. Then \[
\begin{align*}
    \mathbf{U}(\btheta)=\sum_{i \in S_A} \bx_i -\sum_{i \in S_B} d_i^B \pi\left(\mathbf{x}_i, \btheta\right) \bx_i.
\end{align*}
\] and analytical Jacobian is given by \[
\begin{align*} 
\frac{\partial \mathbf{U}}{\partial\btheta} = - \sum_{i \in S_B} d_i^B \pi_i^A\left(\bx_i^{\mathrm{T}} \btheta \right) \left(1 -  \pi_i^A\left(\bx_i^{\mathrm{T}} \btheta \right)\right)\bx_i \bx_i^{\mathrm{T}}.
\end{align*}
\] The second proposed of the smooth function in the literature is
\(\operatorname{h}\left(\bx_i\right) = \bx_i \pi_i^A\left(\bx_i^{\mathrm{T}} \btheta \right)^{-1}\),
for which the \(\operatorname{U}\)-function takes the following form \[
\begin{align}
    \mathbf{U}(\btheta)=\sum_{i \in S_A}  \bx_i \pi_i^A\left(\bx_i^{\mathrm{T}} \btheta \right)^{-1} -\sum_{i \in S_B} d_i^B \bx_i.
\end{align}
\] Generally, the goal is to find solution for following system of
equations \[
\begin{equation*}
    \sum_{i \in S_A} \mathbf{h}\left(\mathbf{x}_i, \btheta\right) = \sum_{i \in S_B} d_i^B \pi\left(\mathbf{x}_i, \btheta\right) \mathbf{h}\left(\mathbf{x}_i, \btheta\right)
\end{equation*}
\] In total, we have six models for this estimation method depending on
the \(\operatorname{h}\)-function and the way propensity score is
parameterized. Let us present all of them.

\hypertarget{logistic-regression-1}{%
\subsection{Logistic regression}\label{logistic-regression-1}}

As the one model for logistic regression is presented above, we have
equation under
\(\operatorname{h}\left(\bx_i\right) = \bx_i \pi_i^A\left(\bx_i^{\mathrm{T}} \btheta \right)^{-1}\)
to consider. Analytical jacobian is given by \[
\begin{align*}
    \frac{\partial \operatorname{U}(\btheta)}{\partial \btheta} = -\sum_{i \in S_A} \frac{1 - \pi_i^A\left(\bx_i^{\mathrm{T}} \btheta \right)}{\pi_i^A\left(\bx_i^{\mathrm{T}} \btheta \right)} \bx_i \bx_i^{\mathrm{T}}.
\end{align*}
\]

\hypertarget{complementary-log-log-regression-1}{%
\subsection{Complementary log-log
regression}\label{complementary-log-log-regression-1}}

For
\(\operatorname{h}\left(\bx_i\right) = \bx_i \pi_i^A\left(\bx_i^{\mathrm{T}} \btheta \right)^{-1}\)
analytical jacobian is given by \[
\begin{align*}
    \frac{\partial \operatorname{U}(\btheta)}{\partial \btheta} = - \sum_{i \in S_A} \frac{1 - \pi_i^A\left(\bx_i^{\mathrm{T}} \btheta \right)}{\pi_i^A\left(\bx_i^{\mathrm{T}} \btheta \right)^2} \exp(\bx_i^{\mathrm{T}} \btheta) \bx_i \bx_i^{\mathrm{T}}
\end{align*}
\] and \(\operatorname{h}\left(\bx_i\right) = \bx_i\) we have \[
\begin{align*}
    \frac{\partial \operatorname{U}(\btheta)}{\partial \btheta} = - \sum_{i \in S_B} \frac{1 - \pi_i^A\left(\bx_i^{\mathrm{T}} \btheta \right)}{\pi_i^B} \exp \left(\bx_i^\mathrm{T} \btheta\right) \bx_i \bx_i^{\mathrm{T}}.
\end{align*}
\]

\hypertarget{probit-regression-1}{%
\subsection{Probit regression}\label{probit-regression-1}}

Similarly, for the probit model, under
\(\operatorname{h}\left(\bx_i\right) = \bx_i \pi_i^A\left(\bx_i^{\mathrm{T}} \btheta \right)^{-1}\)
analyical jacobian is given by \[
\begin{align*}
    \frac{\partial \operatorname{U}(\btheta)}{\partial \btheta} = - \sum_{i \in S_A} \frac{\dot{\pi}_i^A\left(\bx_i^{\mathrm{T}} \btheta \right)}{\pi_i^A\left(\bx_i^{\mathrm{T}} \btheta \right)^2} \bx_i \bx_i^{\mathrm{T}}
\end{align*}
\] and under \(\operatorname{h}\left(\bx_i\right) = \bx_i\) we have

\[
\begin{align*}
    \frac{\partial \operatorname{U}(\partial \btheta)}{\btheta} = - \sum_{i \in S_B} \frac{\dot{\pi}_i^A\left(\bx_i^{\mathrm{T}} \btheta \right)}{\pi_i^B} \bx_i \bx_i^{\mathrm{T}}.
\end{align*}
\]

\hypertarget{population-mean-estimator-and-its-properties}{%
\section{Population mean estimator and its
properties}\label{population-mean-estimator-and-its-properties}}

\[
\begin{equation*}
    \hat{\mu}_{IPW1} = \frac{1}{N} \sum_{i \in S_A} \frac{y_i}{\hat{\pi}_i^{A}}
\end{equation*}
\]

\[
\begin{equation*}
    \hat{\mu}_{IPW2} = \frac{1}{\hat{N}^{A}} \sum_{i \in S_A} \frac{y_i}{\hat{\pi}_i^{A}},
\end{equation*}
\] where \(\hat{N^A} = \sum_{i \in S_A} \hat{d}_i^A\).

\hypertarget{variance-of-an-estimator}{%
\subsection{Variance of an estimator}\label{variance-of-an-estimator}}

Let \(\boldsymbol{\eta} = \left(\mu, \btheta^{T}\right)^{T}\) be the set
of parameters to estimate for inverse probability weighting model. The
estimator
\(\hat{\boldsymbol{\eta}} = \left(\hat{\mu}, \hat{\btheta}^{T}\right)^{T}\)
is the solution of joint estimating equations
\(\boldsymbol{\Phi}_n(\boldsymbol{\eta}) = \bZero\). \[
\begin{equation}
\boldsymbol{\Phi}_n(\boldsymbol{\eta})=\left(\begin{array}{c}
\frac{1}{N} \sum_{i=1}^N\left[\frac{R_i\left(y_i-\mu\right)}{\pi_i^A}+\Delta \frac{R_i-\pi_i^A}{\pi_i^A}\right] \\
\mathbf{U}(\btheta)
\end{array}\right)=\bZero,
\end{equation}
\] where where \(\Delta = \mu\), if \(\hat{\mu} = \hat{\mu}_{IPW1}\) and
\(\Delta = 0\) if \(\hat{\mu} = \hat{\mu}_{IPW2}\).
\(\mathbf{U}(\btheta)\) is objective function corresponding to given way
od estimation. For example in case od pseudo maximum likelihood
estimation, we have gradient corresponding to the choosen link function.
In case of GEE it will be on of considered estimating equations. We have
\(\mathbb{E} \left\{\boldsymbol{\Phi}_n(\boldsymbol{\eta})\right\} = \bZero\)
when
\(\boldsymbol{\eta} = \boldsymbol{\eta}_{0} = \left(\mu_y, \btheta_0^{T}\right)^{T}\).
By applying the first order Taylor expansion around
\(\boldsymbol{\eta}_0\), we further have \[
\begin{equation*}
\hat{\boldsymbol{\eta}}-\boldsymbol{\eta}_0=\left[\boldsymbol{\phi}_n\left(\boldsymbol{\eta}_0\right)\right]^{-1} \boldsymbol{\Phi}_n\left(\boldsymbol{\eta}_0\right)+o_p\left(n_A^{-1 / 2}\right)=\left[E\left\{\boldsymbol{\phi}_n\left(\boldsymbol{\eta}_0\right)\right\}\right]^{-1} \boldsymbol{\Phi}_n\left(\boldsymbol{\eta}_0\right)+o_p\left(n_A^{-1 / 2}\right),
\end{equation*}
\] where
\(\boldsymbol{\phi}_n(\boldsymbol{\eta})=\partial \boldsymbol{\Phi}_n(\boldsymbol{\eta}) / \partial \boldsymbol{\eta}\).
It follows that \[
\begin{equation}
\operatorname{Var}(\hat{\boldsymbol{\eta}})=\left[E\left\{\boldsymbol{\phi}_n\left(\boldsymbol{\eta}_0\right)\right\}\right]^{-1} \operatorname{Var}\left\{\boldsymbol{\Phi}_n\left(\boldsymbol{\eta}_0\right)\right\}\left[E\left\{\boldsymbol{\phi}_n\left(\boldsymbol{\eta}_0\right)\right\}^{\top}\right]^{-1}+o\left(n_A^{-1}\right).
\end{equation}
\] Let us show how to derive this variance-covariance matrix in case of
MLE with logistic regression model. Calculations for the rest of the
models is available in Appendix and you are welcome to take a look at
them. Thus, we have \[
\boldsymbol{\Phi}_n(\boldsymbol{\eta})=\left(\begin{array}{c}
\frac{1}{N} \sum_{i=1}^N\left[\frac{R_i\left(y_i-\mu\right)}{\pi_i^A}+\Delta \frac{R_i-\pi_i^A}{\pi_i^A}\right] \\
\frac{1}{N} \sum_{i=1}^N R_i \boldsymbol{x}_i-\frac{1}{N} \sum_{i \in \mathcal{S}_B} d_i^B \pi_i^A \boldsymbol{x}_i
\end{array}\right)
\] and

\[
\phi_n(\boldsymbol{\eta})=\frac{1}{N}\left(\begin{array}{cc}
-\sum_{i=1}^N\left\{\left(1-\frac{\Delta}{\mu}\right) \frac{R_i}{\pi_i^A}+\frac{\Delta}{\mu}\right\} & -\sum_{i=1}^N \frac{1-\pi_i^A}{\pi_i^A} R_i\left(y_i-\mu+\Delta\right) \boldsymbol{x}_i^{\top} \\
\mathbf{0} & -\sum_{i \in \mathcal{S}_B} d_i^B \pi_i^A\left(1-\pi_i^A\right) \boldsymbol{x}_i \boldsymbol{x}_i^{\top}
\end{array}\right)
\] It can be shown that \[
\left[E\left\{\boldsymbol{\phi}_n(\boldsymbol{\eta})\right\}\right]^{-1}=\left(\begin{array}{cc}
-1 &  \mathbf{b}^{\top} \\
\mathbf{0} & -\left[\frac{1}{N} \sum_{i=1}^N \pi_i^A\left(1-\pi_i^A\right) \boldsymbol{x}_i \boldsymbol{x}_i^{\top}\right]^{-1}
\end{array}\right)
\] where

\[
\mathbf{b}^{\top} = 
\left\{N^{-1} \sum_{i=1}^N\left(1-\pi_i^{\mathrm{A}}\right) \left(y_i-\mu_y + \Delta\right) x_i^{\top}\right\}\left\{N^{-1} \sum_{i=1}^N \pi_i^{\mathrm{A}}\left(1-\pi_i^{\mathrm{A}}\right) \boldsymbol{x}_i x_i^{\top}\right\}^{-1}
\]

\(\operatorname{Var}\left\{\boldsymbol{\Phi}_n\left(\boldsymbol{\eta}_0\right)\right\}\)
can be found using decomposition of
\(\bPhi_n(\boldsymbol{\eta}) = \bA_1 + \bA_2\). Then
\(\operatorname{Var}\left\{\boldsymbol{\Phi}_n\left(\boldsymbol{\eta}_0\right)\right\} = \operatorname{Var}\left(\bA_1\right) + \operatorname{Var}\left(\bA_2\right)\).
Let \[
\mathbf{A}_1=\frac{1}{N} \sum_{i=1}^N\left(\begin{array}{c}
\frac{R_i\left(y_i-\mu\right)}{\pi_i^A}+\Delta \frac{R_i-\pi_i^A}{\pi_i^A} \\
R_i \boldsymbol{x}_i-\pi_i^A \boldsymbol{x}_i
\end{array}\right), \quad \mathbf{A}_2=\frac{1}{N}\left(\begin{array}{c}
0 \\
\sum_{i=1}^N \pi_i^A \boldsymbol{x}_i-\sum_{i \in S_B} d_i^B \pi_i^A \boldsymbol{x}_i
\end{array}\right)
\]

With this division, we have
\(\operatorname{Var}\left\{\boldsymbol{\Phi}_n\left(\boldsymbol{\eta}_0\right)\right\}=\mathbf{V}_1+\mathbf{V}_2\)
where \(\mathbf{V}_1 = Var \left(A_1\right)\) and
\(\mathbf{V}_2 = Var \left(A_2\right)\). \(V_1\) depends only on the
model for propensity score and \(V_2\) on sampling design for
probability sample, both evaluated on
\(\boldsymbol{\eta} = \boldsymbol{\eta}_0\). Finally we have \[
\mathbf{V}_1=\frac{1}{N^2} \sum_{i=1}^N\left(\begin{array}{cc}
\left\{\left(1-\pi_i^A\right) / \pi_i^A\right\}\left(y_i-\mu+\Delta\right)^2 & \left(1-\pi_i^A\right)\left(y_i-\mu+\Delta\right) \boldsymbol{x}_i^{\top} \\
\left(1-\pi_i^A\right)\left(y_i-\mu+\Delta\right) \boldsymbol{x}_i & \pi_i^A\left(1-\pi_i^A\right) \boldsymbol{x}_i \boldsymbol{x}_i^{\top}
\end{array}\right)
\] and \[
\mathbf{V}_2=\left(\begin{array}{ll}
0 & \mathbf{0}^{\top} \\
\mathbf{0} & \mathbf{D}
\end{array}\right)
\] where
\(\mathbf{D}=N^{-2} V_p\left(\sum_{i \in \mathcal{S}_B} d_i^B \pi_i^A \boldsymbol{x}_i\right)\)
and is given by \[
\begin{equation}
{\mathbf{D}}=\frac{1}{N^2} \sum_{i \in \mathcal{S}_{\mathrm{B}}} \sum_{j \in \mathcal{S}_{\mathrm{B}}} \frac{\pi_{i j}^{\mathrm{B}}-\pi_i^{\mathrm{B}} \pi_j^{\mathrm{B}}}{\pi_{i j}^{\mathrm{B}}} \frac{{\pi}_i^{\mathrm{A}}}{\pi_i^{\mathrm{B}}} \frac{{\pi}_j^{\mathrm{A}}}{\pi_j^{\mathrm{B}}} \boldsymbol{x}_i \boldsymbol{x}_j^{\top}
\end{equation}
\] The asymptotic variance for \(\mu_{IPW}\) is the first diagonal
element of matrix \[
\left[E\left\{\boldsymbol{\phi}_n\left(\boldsymbol{\eta}_0\right)\right\}\right]^{-1} \operatorname{Var}\left\{\boldsymbol{\Phi}_n\left(\boldsymbol{\eta}_0\right)\right\}\left[E\left\{\boldsymbol{\phi}_n\left(\boldsymbol{\eta}_0\right)\right\}^{\top}\right]^{-1}.
\]



\end{document}
